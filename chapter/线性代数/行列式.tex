\section{行列式} \label{sec:行列式}
我们预先定义:
\begin{itemize}
    \item $S_n$ 为 $\{1, 2, \ldots, n\}$ 的置换的全体,如 
        $$
        \begin{aligned}
            S_3 =  \{&(1\ 2\ 3), (1\ 3\ 2), (2\ 1\ 3), (2\ 3\ 1), (3\ 1\ 2), \\
                     &(3\ 2\ 1)\}
        \end{aligned}
        $$
    \item 对于置换 $\sigma$,我们令 $\FnSgn(\sigma)$ 为 $\sigma$ 的符号差,对于
        数对 $(i, j)$,如果 $1 \le i < j \le n \land \sigma(i) > \sigma(j)$,那
        么我们称其为一个逆序,记逆序对的数量为 $m$,有:
        $$
        \FnSgn(\sigma) = \begin{cases}
            1,  & m \bmod 2 = 0,     \\
            -1, & \text{otherwises}. \\
        \end{cases}
        $$
        很明显,$\FnSgn((1\ 2\ 3)) = 1$,$\FnSgn((2\ 1\ 3)) = -1$。
\end{itemize}

那么行列式的定义如下:
$$
|\MATID A| = \FnDet(\MATID A) = \sum_{\sigma \in S_n}\FnSgn(\sigma) \prod_{i=1}^n a_{i, \sigma(i)}
$$
我们可以注意到 $S_n$ 中一共有 $n!$ 的求和项。

比如说:
\[
    \begin{vmatrix}    
        a & b \\
        c & d \\
    \end{vmatrix}
    = 
    a d - b c
\]

两向量如果共线,那么其构成的行列式也为零。

\subsection{性质} \label{subsec:行列式的性质}
当且仅当 $|\MATID A| = 0$ 的时候,$\MATID A \vec v = \vec 0$ 才存在一个平凡解。
