\section{特征向量} \label{sec:特征向量}
给定一个矩形 $\MATID A$,我们定义它的特征向量 $\vec{x}$ 是存在一个常数 $\lambda$ 使得
:\[ \MATID A \vec{x} = \lambda \vec{x} \]其中 $\lambda$ 是特征值。

对于二阶矩阵而言,一般都有两个特征向量。我们可以用特征向量的定义来做。

比如我们打算求下列矩阵的特征向量:\[
    \begin{bmatrix}
        1 & 1 \\ 1 & 0
    \end{bmatrix}
\]

那么为了避免奇点,我们搞一个长度为 $1$ 的向量,设 $\vec x = (\sin \theta, \cos
\theta)$。

根据定理,我们有:\[
    \begin{bmatrix}
    1 & 1 \\
    1 & 0
    \end{bmatrix} \begin{bmatrix}
    \sin \theta \\ \cos \theta
    \end{bmatrix}
    =
    \begin{bmatrix}
    \sin \theta + \cos \theta \\ \sin \theta
    \end{bmatrix}
\]

因为特征向量被其矩阵相乘后方向不变,那么下行列式的值为 $0$(见 \ref{sec:行列式}
章):\[\begin{aligned}
    & \begin{vmatrix}
        \sin \theta & \sin \theta + \cos \theta \\
        \cos \theta & \sin \theta \\
    \end{vmatrix} \\ = & {}\sin^2 \theta - \sin \theta \cos \theta - \cos^2 \theta
\end{aligned}\]

即 $\tan^2 \theta - \tan \theta - 1 = 0$,解得:\[
    \tan \theta = {1 \pm \sqrt 5 \over 2}
\]

那么我们就可以求出 $(\sin\theta, \cos\theta)$ 了。

它可以用来求取矩阵的幂,见 \ref{sec:矩阵的幂} 章。

