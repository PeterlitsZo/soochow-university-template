\section{最短非子序列} \label{sec:最短非子序列}
给定一个字符集 $\Sigma$ 和一个串 $S$,我们需要求它的最短非子序列。

这并不难,我们贪心地解即可:子序列的每一个字符,都在前一个匹配的基础上,尽可能地
向后匹配。

比如若 $\Sigma = \{a, b\}$,那么字符串 $S = \text{\cmd{"abaabb"}}$ 对应的最短
非子序列为 \cmd{"bba"},其中第一个字符应当是 $b$,因为它匹配的是第二位,比 $a$匹
配得远。

对于一个而言,我们很容易就可以解出来了,但是对于一个串的子串而言,我们需要预处理
来降低复杂度。答案就是倍增(倍增见 \ref{sec:倍增} 章节)来跳跳跳。

参考代码:
\begin{Cpp}
static char S[LEN];
static int MP[30],JP[LEN][20];

// 使用 MP 来得到基本的数据(JP[x][0])。
fill(MP,MP+30,n+1);
JP[n+1][0]=n+1;
for(int i=n;i>=0;i--){
  JP[i][0]=*max_element(MP,MP+m);
  MP[S[i]-'a']=i;
}

// 得到 JP[x][y]。
for(int i=1;i<20;i++){
  JP[n+1][i]=n+1;
  for(int j=0;j<=n;j++){
    JP[j][i]=JP[JP[j][i-1]][i-1];
  }
}

// 查询
int q;scanf("%d",&q);
for(int i=0;i<q;i++){
  int l,r;scanf("%d%d",&l,&r);
  int bef=l-1,ans=0;
  // 退出循环的标志位。
  bool flg=0;
  while(true){
    for(int i=0;i<20;i++){
      if(JP[bef][i]>r) {
        // 当 i==0 且 JP[bef][i]>r,就该退出了。
        if(i) ans+=1<<(i-1),bef=JP[bef][i-1];
        else flg=1,ans+=1; 
        break;
      }
    }
    if(flg)break;
  }
  printf("%d\n",ans);
}
\end{Cpp}
