\section{哈希} \label{sec:哈希}
哈希可以将字符串映射到一个整数集合上。

一般而言我们将字符串类比成多项式:如 $$f_{S = \text{\ttfamily "abc"}}(x) =
\FnOrd(\mathtt{'a'})x^2 + \FnOrd(\mathtt{'b'})x + \FnOrd(\mathtt{'c'})$$

那么如果令 $x > \#\Sigma$,且 $\FnOrd$ 能将 $\Sigma$ 中的所有值单射到 $R$,其中
$R = \{1, 2, \ldots, \#\Sigma\}$, 那
么我们可以说如果两个字符串不同,当且仅当它们的 $f$ 多
项式不同,也等价于对于 $x > \#\Sigma$ 时,多项式的对应的值不同,即 $S \ne S' \iff
f_S(x) \ne f_{S'}(x),\, \forall x > \#\Sigma$。

因为一般而言 $\#\Sigma = 26$,我们可以取 $29$ 和 $31$。而因为整形类型不是封闭的
,可能会有溢出,我们需要取一个足够大的素数模使得它是一个模下域。

因此有代码:
\begin{Cpp}
namespace HS{
  constexpr int LMT=/*字符串的长度*/,
    MOD=(int)1e9+7,BAS=29;
  ll POW[LMT],IPOW[LMT];
  inline void init(){
    POW[0]=1;
    for(int j=1;j<LMT;j++){
      POW[j]=POW[j-1]*BAS%MOD;
    }
    IPOW[LMT-1]=fpow(POW[LMT-1],MOD-2,MOD);
    for(int j=LMT-2;j>=0;j--){
      IPOW[j]=IPOW[j+1]*BAS%MOD;
    }
  }
  inline int map(char ch){
    return ch-'A'+1;
  }
  inline void push(int&A,char ch){
    A=(A*POW[1]+map(ch))%MOD;
  }
  inline void push(int A[],char S[],int slen){
    int tmp=0;
    for(int i=0;i<slen;i++){
      push(tmp,S[i]);A[i]=tmp;
    }
  }
  inline int query(int A[],int b,int e){
    return (A[e]-(b?A[b-1]:0)*POW[e-b+1]%MOD
      +MOD)%MOD;
  }
}
\end{Cpp}

有时候,可以使用 $131$ 这个数字,用溢出来取模,可能也是可以的。小狮子说,字符集
大小啥的不重要!硬冲就行。因为出题人猜不到我们选的基数是什么。

有的时候会因为哈希冲突而过不了,这个时候我们需要使用大一点的东西,比如
\verb|array|。

\begin{Cpp}
namespace HS{
  constexpr int CNT=2,LMT=LEN;
  constexpr int MOD[CNT]={
    (int)1e9+7,(int)1e9+9};
  constexpr int BAS[CNT]={29,31};
  ll POW[CNT][LMT],IPOW[CNT][LMT];
  inline void init(){
    for(int i=0;i<CNT;i++){
      POW[i][0]=1;
      for(int j=1;j<LMT;j++){
        POW[i][j]=POW[i][j-1]*BAS[i]%MOD[i];
      }
      IPOW[i][LMT-1]=fpow(POW[i][LMT-1],
        MOD[i]-2,MOD[i]);
      for(int j=LMT-2;j>=0;j--){
        IPOW[i][j]=IPOW[i][j+1]*BAS[i]%MOD[i];
      }
    }
  }
  typedef array<int,CNT> RES;
  inline int map(char ch){
    return ch-'A'+1;
  }
  inline void push(RES&A,char ch){
    for(int i=0;i<CNT;i++)
      A[i]=(A[i]*POW[i][1]+map(ch))%MOD[i];
  }
  inline void push(RES A[],char S[],int slen){
    RES tmp=RES();
    for(int i=0;i<slen;i++){
      push(tmp,S[i]);A[i]=tmp;
    }
  }
  inline void print(RES&A){
    printf("<");
    for(int i=0;i<CNT;i++)printf("%d,",A[i]);
    printf(">");
  }
  inline RES query(RES A[],int b,int e){
    RES res=RES();
    for(int i=0;i<CNT;i++)
      res[i]=(A[e][i]
        -(b?1ll*A[b-1][i]*POW[i][e-b+1]%MOD[i]:0)
        +MOD[i])%MOD[i];
    return res;
  }
}

// 使用 HS::RES hs = HS::RES() 来初始化哦。
\end{Cpp}
