\chapter{代数和抽象代数}

\section{韦达定理}
对于 $ax^2 + bx + c = 0$ 的二次多项式而言,其根的关系满足韦达定理:$x_1 + x_2 =
- {b \over a}$,和 $x_1 x_2 = {c \over a}$。这个可以用 $$x_1, x_2 = {-b \pm
\sqrt{b^2 -4ac} \over 2a}$$ 简单的推出来。

\section{陪集}
对于群 $G$,如果 $H$ 为 $G$ 的子群,那么对于 $g \in G$,我们认为 $gH$ 是 $G$ 的
左陪集,$Hg$ 是 $G$ 的右陪集。

\section{代数系统}
代数系统是对相关的代数系统做的一个简单的抽象。一般而言,代数系统由操作的集合 $S$
和在 $S$ 上的运算构成,我们一般把运算表示为 $\circ$。运算的特殊性质有:封闭性,
结合性,可交换性,幂等性,可分配性(这个需要两个运算),等等。在代数系统的 $S$ 集合中
,可能有一些特殊的元素,比如幺元(或者单位元)或者零元,还有逆元。

\subsubsection{半群}
如果代数系统 $\langle S, \circ \rangle$ 中运算 $\circ$ 满足结合性,那么我们说这
个代数系统是半群。

\subsubsection{独异点}
一个有幺元的半群,我们称它是独异点。

\subsubsection{群}
每一个元素都有逆元的独异点,我们叫它是群。

\subsubsection{环}
代数系统可以不只带一个运算。如果代数系统 $\langle A, +, *\rangle$ 满足:
\begin{itemize}
    \item $\langle A, +\rangle$ 是一个阿贝尔群。
    \item $\langle A, *\rangle$ 是一个半群。
    \item 运算 $*$ 对于运算 $+$ 而言是可分配的。
\end{itemize}

那么我们称这个代数系统是一个环。

