\section{斯大林数}
\label{sec:斯大林数}

第三类斯大林数见章节 \ref{sec:拉赫数}。

斯大林数分为两种。一般用来描述如何把元素分组的问题。其中一般而言第二类斯大林数更
为普遍。

\subsection{第二类斯大林数}
第二类斯大林数解决的问题是:给定 $n$ 个互不相同的球,放到 $k$ 个相同的盒子里(非
空),有多少种情况数,我们把它记为 $\CmdStrII{n}{k} = S(n, k)$。

很明显有:\[
\begin{gathered}
    k = n \lor k = 1 \Rightarrow \CmdStrII{n}{k} = 1,\\
    k > n \lor k = 0 \Rightarrow \CmdStrII{n}{k} = 0, \\
\end{gathered}
\]

基本式如上,而递归式如下:\[
    \CmdStrII{n}{k} = \CmdStrII{n-1}{k-1} + k \cdot \CmdStrII{n-1}{k}
\]

对于行的第二类斯大林数,我们一般把它写成 \[
    \CmdStrII{n}{k} = {1 \over k!} \sum^{k}_{i=0} (-1)^i {k \choose i} (k - i)^n
    % \begin{StrII}n \\ \fbox{k}\end{StrII} = \sum^m_{i=0} {i^n \over i!} {(-1)^{m-i}
    % \over (m-1)!}
\]

而对于列而言我们有 $\begin{StrII}\fbox{n} \\ k\end{StrII}$:\[
    S(x) = \sum_{i=0}^{\infty} \begin{StrII}i \\ k\end{StrII} x^i
\] \[
    S(x) = {x^m \over \prod^{m}_{i=1} (1 - ix)}
\]

这个需要分治 NTT。

\subsection{第一类斯大林数}
第二类斯大林数解决的问题是:给定 $n$ 个互不相同的球,放到 $k$ 个圆排列的盒子中。
我们把它记作 $\CmdStrI{n}{k}$:\[
    \CmdStrI{n}{k} = \CmdStrI{n-1}{k-1} + (n-1)\CmdStrI{n-1}{k}
\]

