\section{排列组合}
排列数 $A_n^k$ 有:\[
    A_n^k = {n! \over (n-k)!}
\]

而组合数 $C_n^k = {n \choose k}$ 有:\[
    {n \choose k} = {n! \over k! (n-k)!}
\]



\subsection{模为素数时的打表法}
我们可以使用下列函数打表求出排列数和组合数(要求模 $M$ 为一个素数,以保证有乘法
逆元)(快速幂代码见章节 \ref{sec:快速幂}):
\begin{Cpp}
constexpr MOD=/* 模 */;
constexpr LEN=/* 排列组合的参数范围,必须 << MOD */;
ll fac[LEN],inv[LEN];

ll fpow(ll b,ll p);

// 初始化。别忘了。
void init(){
  // etc...
  fac[0]=1;
  for(int i=1;i<LEN;i++)
    fac[i]=fac[i-1]*i%MOD;
  inv[LEN-1]=fpow(fac[LEN-1],MOD-2);
  for(int i=LEN-2;i>=0;i--)
    inv[i]=inv[i+1]*(i+1)%MOD;
}

// 可选:求排列数
ll A(int N,int m){
  if(m>N||m<0)return 0;
  return fac[N]*inv[N-m]%MOD;
}

// 可选:求组合数
ll C(int N,int m){
  // 如果有 A 函数的话,直接:
  // return A(N,m)*inv[m]%MOD;
  if(m>N||m<0) return 0;
  return fac[N]*inv[m]%MOD*inv[N-m]%MOD;
}
\end{Cpp}



\subsection{卢卡斯}
如果 $n$ 之类的范围比较大而 $M$(模)较小,可以参考卢卡斯定理或者扩展卢卡斯定理
(见章节 \ref{sec:卢卡斯})。



\subsection{杨辉三角打表法}
如果 $n$ 之类的范围比较小,而 $M$ 随意的话,我们也可以打表,不过是杨辉三角,即:
\[
    {n \choose k} \bmod M = \left({n - 1 \choose k - 1} + {n \choose k -
    1}\right) \bmod M
\]

虽然都是打表,不过这个是二维表,前面的代码是一维的,所以虽然不再要求 $M$ 是模,
但是也让 $n$ 的范围变小了。

\TODOtag 记得改成统一的风格。
别人的板子:
\begin{Cpp}
typedef long long lld;
const int maxn = 1000+10;

lld C_arr[maxn+10][maxn+10];

void C_init(int n, int pr) {
	for (int i = 0; i <= n; i++) {
		C_arr[i][0] = C_arr[i][i] = 1;
		for (int j = 1; j < i; j++)
			C_arr[i][j] = (C_arr[i - 1][j - 1]
                + C_arr[i - 1][j]) % pr;
	}
}

lld C(int n, int m) {
	return C_arr[n][m];
}
\end{Cpp}



\subsection{性质}
组合数学有着一些独有的性质:

选择状态取反:\[
    {n \choose k} = {n \choose n - k}
\]

定义推导出来的玩意:\[
    {n \choose k} = {n! \over k! (n - k)!} = {n \over k} {n - 1 \choose k - 1}
\]

加法原理,分为选自己和不选自己两种情况:\[
    {n \choose k} = {n - 1 \choose k - 1} {n - 1 \choose k}
\]

瞎鸡儿乱选,那么每一元素就是两种独立的状态,乘法原理有(章节 \ref{sec:二项式定%
理} 二项式定理中对应着 $a = b = 1$):\[ \sum_{i} {n \choose i} = 2^n \]

只选奇数或者只选偶数:\[
    \sum_{i \bmod 2 = 0} {n \choose i} = 2^{n - 1}
\]

加加减减地选择,可以用二项式定理(见章节 \ref{sec:二项式定理})来推:\[
    \sum^{n}_{i = 0} (-1)^i {n \choose i} = [n = 1]
\]

拆组合数:\[
    \sum_{i=0}^{k} {n \choose i} {k \choose k - i} = {n + k \choose k}, \qquad (n
    \ge k)
\] \[
    \sum_{i=0}^{n} {n \choose i}^2 = {2n \choose n}
\]

带权和的一个例子:\[
    \sum_{i=0}^{n} i{n \choose i} = n2^{n - 1}
\] \[
    \sum_{i=0}^{n} i^2{n \choose i} = n(n + 1)2^{n - 2}
\]

恒等式证明中较常用:\[
    \sum_{l=0}^{n} {l \choose k} = {n + 1 \choose k + 1}
\]

和斐波那契数有关的式子:\[
    \sum_{i = 0}^{n} {n - i \choose i} = F_{n + 1}
\]
