\section{排列组合}
排列数 $A_n^k$ 有:\[
    A_n^k = {n! \over (n-k)!}
\]

而组合数 $C_n^k = {n \choose k}$ 有:\[
    {n \choose k} = {n! \over k! (n-k)!}
\]

我们可以使用下列函数求出排列数和组合数(快速幂代码见章节 \ref{sec:快速幂}):
\begin{Cpp}
ll fac[LEN],inv[LEN];

ll fpow(ll b,ll p);

void init(){
  fac[0]=1;
  for(int i=1;i<LEN;i++)
    fac[i]=fac[i-1]*i%MOD;
  inv[LEN-1]=fpow(fac[LEN-1],MOD-2);
  for(int i=LEN-2;i>=0;i--)
    inv[i]=inv[i+1]*(i+1)%MOD;
}

ll A(int N,int m){
  if(m>N||m<0)return 0;
  return fac[N]*inv[N-m]%MOD;
}

ll C(int N,int m){
  // 如果有 A 函数的话,直接:
  // return A(N,m)*inv[m]%MOD;
  if(m>N||m<0) return 0;
  return fac[N]*inv[m]%MOD*inv[N-m]%MOD;
}
\end{Cpp}



\subsection{捆绑法}

把一些元素捆绑在一起以视为一个独立的元素。

比如有 $5$ 个男生和 $3$ 个女生。女生都要在一起。问有多少种不同的排队情况。那么有
:$A_6^6 A_3^3$



\subsection{插板法}
\label{subsec:插板法} \label{subsec:插空法}

如果在 $m$ 个球中搞到 $n$ 个盒子中,盒子两两之间都是不同的,那么:
\begin{itemize}
    \item 如果不允许为空,那么解是 $m - 1 \choose n - 1$。这个解是因为 $m$ 个球
        ,不包含两端有 $m - 1$ 个空,然后选择 $n - 1$ 个隔板。隔出来了之后就是
        $n$ 个盒子了。
    \item 如果允许为空,那么解是 $m + n - 1 \choose n - 1$。这个解是因为它插入了
        虚拟球,一共有 $n$ 个盒子,那么我们在里面插入 $n$ 个虚拟球。选出来了之后
        指定第一个是虚拟球去掉即可。
\end{itemize}

