\section{十二重}
十二重是组合数学中最经典的十二问。嗷:
\begin{enumerate}
    \item 球异盒异:随便乱放,$m^n$。
    \item 球异盒异 \& 每一个盒子最多一个球:${m \choose n} n!$。
    \item 球异盒异 \& 每一个盒子最少一个球,考虑容斥定理(见章节 \ref{sec:容斥%
        定理})。瞎鸡儿乱放可能会又空箱子,用高贵的容斥剪掉剪掉就可以了。\[
            \sum^m_{i=0} (-1)^i {m \choose i} (m - i)^n
        \]
        其中 $i$ 表示了一共有多少个真的空盒子。而那些不在 $i$ 对应的真空盒子中的
        假空盒子,就需要下面的若干个 $i$ 来剪掉了。

    \item 球同盒异:插板法,见章节 \ref{subsec:插板法}:$n + m - 1 \choose m -
        1$。
    \item 球同盒异 \& 每一个盒子最多一个球:$m \choose n$。
    \item 球同盒异 \& 每一个盒子最少一个球:$n - 1 \choose m - 1$。
\end{enumerate}
