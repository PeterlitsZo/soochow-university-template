\section{图上博弈}
\subsection{树上博弈}
在一个树上移动,不能移动到已经访问过的点的题,这个题叫做树上博弈。

一般而言,如果一个树退化到一个链,而一开始点在端点上,那么这个博弈就退化成了最多
取一个的巴什博弈(见 \ref{sec:巴什博弈} 章),那么也就是说,奇数个点对应着的是 P
状态,而偶数个点对应着的是 N 状态。

而一开始点不在端点上,在链的中间,那么它就可以看成两个链拼起来,只要存在一个链是
N 状态,那么我们就会选择它(也就是说,可以移动到一个 P 状态上)。

而更复杂的,除了根没有其他分支的和这个情况也相同。

更加复杂的,我们考虑一个普通的图,只要有一个子图对应着的是 N 状态(也就是说可以
移动到一个 P 状态上),那么这个子图就是 N 状态的。
