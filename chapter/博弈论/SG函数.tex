\section[基础博弈:SG 函数]{基础博弈:SG 函数\footnote{参考了文章
\url{https://blog.csdn.net/strangedbly/article/details/51137432}}}
组合博弈论的问题是一种可以建立在 P/N 状态上从而抽象成图的问题。总的来说,它的性
质为:
\begin{enumerate}
    \item 如果它是一个终端节点(因此无法移动到其他节点上)的话,那么它是一个 P
        局面。我们可以用 P 来表示。
    \item 如果它可以移动到 P 节点的节点是一个 N 节点。
    \item 如果所有的操作到达的节点都是 N 节点,那么它就是一个 P 节点。
\end{enumerate}

根据这个性质,简单的枚举一下所有的情况,使用 DP 来做的话(或者记忆化搜索,都是一
个东西吧),简简单单就搞出来了。

这样子,好像 SG 函数最后要么为 0(表示 P 节点),要么为 1(表示 N 节点)。那么除
了这种表示方法,还有其他的表示方法吗?有的,一般来说,我们使用 $\mex$ 来表示 SG
函数。好家伙,为啥不用简简单单的 0/1 而使用 $\mex$ 函数呢?

那我们考虑下,现在貌似所有的博弈论的题目都可以抽象成一个有向无环图上移动棋子的问
题,那么如果这个博弈论的题目不是只有一个棋子,而是多个棋子呢?那么这种时候,这种
SG 函数的意义是什么?

那么我们就要追寻 $\mex$ 函数的意义了。对于 $\mex$ 对应的值为 $k$ 的情况下,我们
可以断言通过这个节点,我们可以到达 $0$,$1$ 一直到 $k-1$ 节点。我的天啊,在多个
石子的情况下,我们可以把它抽象成一个 Nim 游戏。Nim 游戏也是有着多个棋子,并且每
次只能移动一个棋子。如果选择了一个堆的话,那么我们可以把它的值从 $k$ 变成 $0$,
$1$ 一直到 $k-1$。太好了,看起来它们的确是同构的,那么我们只要保证所有的 SG 函数
搞起来它的异或值是 $0$ 就可以了。换言之,如果所有棋子对应的 SG 函数值为 $0$,那
么它就是一个 P-position。

这样子,无论这些棋子在一个无向图上移动,还是说在多个有向图上分别移动,其结论也不
会发生变化。

结论:
\begin{enumerate}
    \item 对于一个由多个子游戏组合而成的游戏 $G$ 而言。我们有 $\FnSG(G) = \FnSG(G_0)
        \oplus \FnSG(G_1) \oplus \ldots \oplus \FnSG(G_{n-1})$。这样子我们就可以构造
        一个 Nim 游戏。
    \item 对于子游戏能用经典模型搞出来的话,其 SG 函数应该也不算难。
        \begin{enumerate}
            \item 比如说,经典的巴什博弈(一堆中,每次取 $1$ 到 $m$ 个),它的 
                SG 函数为:$\FnSG(n) = n \bmod (m + 1)$。
            \item 如果可以选择任意步的话,那么 $\FnSG(n) = n$。
            \item 如果是一系列不连续的话,那么我们要不然分析加找规律($\FnSG$ 的循
                环节对吧?),要不然见 \ref{sec:SG 函数的打表} 章中的打表,要不
                然见 \ref{sec:SG 函数的 DFS} 章中的 DFS 来求解。
        \end{enumerate}
\end{enumerate}

\subsection{SG 函数的打表}\label{sec:SG 函数的打表}
SG 函数的定义蛮简单的,就像 DP 问题一样,很容易就能得到它的公式了,打表,不错,
时间复杂度的确可能会比较高,但有时候不打表就不行嗷。一般而言打表得到 SG 函数(其
实是得到了一个数组来表示这个函数,来表示一定范围内整数到整数的映射)。

\begin{Cpp}
// f 用来表示可以取走的石子的个数。
//    比如说,每一次我们只能取走 1,5,7 个石子,
//    那么我们需要令 f = [1, 5, 7, INF]。
// sg 即我们利用 getSG 来得到的 SG 函数值。
// hash 用来求 mex,每一次都需要清空哦。
int f[N], sg[N], hash[N];

void get_sg() {
    int i, j;
    // 初始化 sg 并得到 sg 数组。
    memset(sg, 0, sizeof(sg));
    for(i = 0; i < N; i++) {
        memset(hash, 0, sizeof(hash));
        // 将能走到的状态的 sg 加入到 hash 中
        // 不过不能使用 f 的话,我们可以使用公式
        // 来描述。
        for(j = 1; f[j] <= i; j++)
            hash[sg[i - f[j]]] = 1;
        // sg 即对 hash 求 mex
        for(j = 0; j < N; j++) {
            if(hash[j] = 0) {
                sg[i] = j;
                break;
            }
        }
    }
}
\end{Cpp}

可以看到,如果我们使用打表的方法来得到 SG 函数的话,那么我们的时间复杂度为 $
\FnO(N \cdot M)$。其中 $M$ 表示了对于每一个节点而言,平均能抵达多少个节点。

\subsection{SG 函数的 DFS}\label{sec:SG 函数的 DFS}

但是有的时候,打表好像不能在指定时间内搞出来,那么我们可以使用 DFS 来把他搞出来
。这个时候我们可以使用:
\begin{Cpp}
// sg 默认初始化为全为 -1。
int s[110], sg[SG_L], n;

void init() {
    // init 的其他代码...

    for(int i = 0; i < SG_L; i ++)
        sg[i] = -1;
}

int sg_dfs(int x) {
    int i;
    if(sg[x] != -1)
        return sg[x];
    // vis 数组记录能到达的节点的 SG 值。
    bool vis[110];
    memset(vis, 0, sizeof(vis));
    for(int i = 0; i < 100; i ++) {
        // 现在的节点是 x,而能到达的节点是 x - s[i]。
        if(x >= s[i]) {
            sg_dfs(x - s[i]);
            vis[sg[x - s[i]]] = 1;
        }
    }
    // 求出答案并保存在 e 中。和 SG 函数定义的一
    // 样,求出 mex 即可。
    int e;
    for(int i = 0; ; i++) {
        if(!vis[i]) {
            e = i;
            break;
        }
    }
    return sg[x] = e;
}
\end{Cpp}

一般情况下建议打表。如果打表过不了再搞 DFS 做法。

\subsection{SG 的一些优化}
SG 函数默认而言是 $\mex$ 值。但是 $\mex$ 有时候是没有啥用的。那么我们只要使用
0/1 来表示就已经足够了。

如果只是 0/1 的话,我们可以使用类似埃氏筛的做法来求出所有的 SG 值。

对于 0 的东西,那么就是 P-postion,我们只要找到它对应的所有 N-postion,并令他为
1 即可。对于 1 的东西,那么就是 N-postion,对应的无论是啥,我们不动它即可。

比如我们的操作是拿走 $1$ 个石子直到 $m$ 个石子,那么我们可以搞到 SG 函数:

\begin{Cpp}
void sg(int x) {
    // ohno,默认的 P-postion,令它对应的 N-
    // position 的 SG 函数为 1。
    if(sg[x] == 0) {
        for(int k = 1; k < m && x + k < SG_L;
            k ++) {
            sg[x + k] = 1;
        }
    }
}
\end{Cpp}
