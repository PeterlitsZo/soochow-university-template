\section{基础博弈:斐波那契博弈}
一堆石子有 $n$ 个,先取者可以取任意多个,但是不能取完,也不能不取,石子最少有
$2$ 个,下一个人取石子的上限不能超过上次取石子数的两倍。给定石子数 $x$,求先手是
否能必赢?

见表 \ref{tab:斐波那契},我们可以先来看看 $1 \sim 20$ 中哪些是必输态(L),哪些
是必赢态(W)。

\begin{table*}
\centering
\begin{tabular}{lllllllllllllllllll}
    \toprule
    2 & 3 & 4 & 5 & 6 & 7 & 8 & 9 & 10 & 11 & 12 & 13 & 14 & 15 & 16 & 17 & 18 &
    19 & 20 \\
    L & L & W & L & W & W & L & W &  W &  W &  W &  L &  W &  W &  W &  W &  W &
    W &  W  \\
    \bottomrule
\end{tabular}
\caption{斐波那契博弈的前 20 种情况}
\label{tab:斐波那契}
\end{table*}

我们敏锐的发现:所有的必输态好像构成了一个斐波那契数列,即 $2, 3, 5, 8, 13,
\ldots$。下面是有关的证明:

我们假设前面的斐波那契数 $f_{n-2}$ 和 $f_{n-1}$ 是必输态,现在我们需要证明 $f_n$
也是一个必输态。

\begin{enumerate}
    \item 如果先行者取 $x \geq f_{n-2}$,因为斐波那契数列的性质,所以说有 $2x
        \geq 2f_{n-2} \geq f_{n-1} \geq f_n - x$,故先行者必输。

    \item 如果先行者取 $x < f_{n-2}$,我们不妨将 $f_n$ 的堆看为两个堆:$f_{n-2}$
        的堆和$f_{n-1}$ 的堆。我们知道无论先行者取的 $x < f_{n-2}$,最后肯定是后
        行者取完$f_{n-2}$ 的堆,只剩下 $f_{n-1}$ 的堆,而 $f_{n-1}$ 也是必输态,
        故先行者必输。
\end{enumerate}

那么对于不是斐波那契的堆就是必赢态吗?我们可以利用齐肯多夫定理,将堆的模 $x
= f_{a_1} + f_{a_2} + \ldots + f_{a_n}$,因为齐肯多夫定理,任意两个项
$f_{a_i}$ 和 $f_{a_j}$ 中都不连续,即有 $f_{a_1} > 2f_{a_j}$,那么我可以首先干掉
最小的那一个,然后让对方面临多个斐波那契且对方无法把最小的那一个干翻,故对方必输
,而己方必赢。

