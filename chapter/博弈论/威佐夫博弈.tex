\section{威佐夫博弈}
威佐夫博弈扩展到两个堆上,且限制也不同了:每个人可以从一个堆上去任意多个,也可以
在两个堆上取得相同的任意多个。先取完者获胜。求取最后他到底必赢还是必输。

奇异局势为 $(0,0)$,$(1,2)$,$(3,5)$,$(4,7)$,$\ldots$,其中第 $i$ 个奇异局势第
一分量为 $0\ldots i-1$ 的奇异局势中所有分量的 $\mex$ 值,而第二分量为第一分量 +
$i$。

以上是递归形式的威佐夫博弈结论,但是根据贝亚蒂定理,我们可以推导出 $\frac 1\phi
+ \frac 1{\phi^2} = 1 \Rightarrow \lfloor\phi \cdot n\rfloor \text{与} \lfloor
\phi^2 \cdot n\rfloor \text{构成了对正整数的划分}$。又因为 $\phi^2 = \phi + 1$,
恰好满足了第二个条件。

代码应该如下:
\begin{Cpp}
// 0: lost, 1: win
int WZF(int a,int b){
    if(a>b)swap(a,b);
    int k=b-a;
    return floor((1+sqrt(5))/2*k)==a;
}
\end{Cpp}

虽然都说 SG 函数可以用来描述博弈论题目,但是并不总是有效。


