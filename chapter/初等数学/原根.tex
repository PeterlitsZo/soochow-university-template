\section{原根} \label{sec:原根}
如果对于数 $m$ 而言,有另一个数 $a$ 满足 $\gcd(a,m)=1$ 并且 $\delta_m(a) =
\varphi(m)$,那么我们就说 $a$ 是 $m$ 的原根。

\subsection{原根判定}
令 $m \ge 3$,对于满足 $\gcd(a, m) = 1$ 的 $a$ 而言,$a$ 是 $m$ 的原根,当
且仅当对于 $\varphi(m)$ 的每一个素因数 $p$,均有 $a^{\varphi(m) \over p}
\not\equiv 1 \pmod m$。

\subsection{原根个数}
若 $m$ 有原根,那么它原根的个数为 $\varphi(\varphi(m))$。

\subsection{原根存在定理}
一个数存在原根,当且仅当 $m = 2, 4, p^\alpha, 2p^\alpha$,其中 $p$ 是奇素数,
$\alpha \in \mathbb{N}^*$。

\subsection{暴力寻找原根}
王元于 1959 年证明了 $m$ 原根的上界是 $m^{0.25}$。那么如果我们希望寻找 $m$ 的原
根的时候,暴力求解即可。
