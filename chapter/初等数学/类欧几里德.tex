\section{类欧几里德} \label{sec:类欧几里德}
类欧几里德主要是解决方式和欧几里德的方法高度相似。它求解的问题和数论分块所需要解
决的问题近似,但是方法却大不相同。

见 \ref{sec:数论分块} 章节中,数论分块解决形如下式的问题:
$$
\sum^{n}_{i=1} \lfloor{n \over i}\rfloor
$$

而类欧几里德需要解决的问题形如:
$$
\sum^{n}_{i=0} \lfloor{ai + b \over c}\rfloor
$$

我们注意到,当 $a \ge c$ 或者 $b \ge c$ 的时候,我们可能可以将它们拆开成对于 $c$
的倍数和余数,从而简化运算过程,有:
$$
\small
\begin{aligned}
    {} & f(a, b, c, n) \\
    = {} & \sum^{n}_{i=0} \lfloor{ai + b \over c}\rfloor \\
    = {} & \sum^{n}_{i=0} \left\lfloor{(\lfloor{a \over c}\rfloor c + a \bmod c)i +
        (\lfloor{b \over c}\rfloor c + b \bmod c) \over c}\right\rfloor \\
    = {} &{n^2 (n + 1) \over 2} \lfloor{a \over c}\rfloor + (n + 1) \lfloor{b \over c}\rfloor \\ 
    {} & + \sum^{n}_{i=0} \lfloor{(a \bmod c) i + (b \bmod c) \over c}\rfloor
\end{aligned}
$$

所以有:$f(a, b, c, n) = {n^2 (n + 1) \over 2} \lfloor{a \over c}\rfloor + n
\lfloor{b \over c}\rfloor  + f(a \bmod c, b \bmod c, c, n)$。

其他情况下,我们考虑贡献合并,就像莫比乌斯反演一样:
$$
\small
\begin{aligned}
    {} & f(a, b, c, n) \\
    = {} & \sum^{n}_{i=0} \lfloor{ai + b \over c}\rfloor \\
    = {} & \sum^{\lfloor{an + b \over c}\rfloor - 1}_{j=0} \sum_{i=0}^{n} \left[j <
        \left\lfloor{ai + b \over c}\right\rfloor\right] \\
    = {} & \sum^{\lfloor{an + b \over c}\rfloor - 1}_{j=0} \sum_{i=0}^{n} \left[
        j + 1 \le {ai + b \over c}\right] \\
    = {} & \sum^{\lfloor{an + b \over c}\rfloor - 1}_{j=0} \sum_{i=0}^{n}
        \left[i \ge {jc + c - b \over a}\right] \\
    = {} & \sum^{\lfloor{an + b \over c}\rfloor - 1}_{j=0} \sum_{i=0}^{n}
        \left[i > \left\lfloor{jc + c - b - 1 \over a}\right\rfloor\right] \\
    = {} & \sum^{\lfloor{an + b \over c}\rfloor - 1}_{j=0} \left(n - \left\lfloor{jc
        + c - b - 1 \over a}\right\rfloor\right) \\
    = {} & \lfloor{an + b \over c}\rfloor n - \sum^{\lfloor{an+b\over
        c}\rfloor-1}_{j=0} \lfloor{jc + c -  b - 1 \over a}\rfloor
\end{aligned}
$$

我们注意到 $f(c, c - b - 1, a, \left\lfloor{an + b \over c}\right\rfloor - 1)$
即为后式。

我们注意到 $a$ 和 $c$ 换了位置。这说明了我们可以再取模,再递归,从而是一个类辗转
相除的过程。

相关代码如下:
\begin{Cpp}
// => sum^n_{i=0} floor((ai + b) / c)
int f(int n,int a,int b,int c){
  if(a>=c||b>=c)
    return a/c*n*(n+1)/2+b/c*(n+1)
            + f(n,a%c,b%c,c);
  if((a*n+b)/c)
    return (a*n+b)/c*n-f((a*n+b)/c-1,c,c-b-1,a);
  return 0;
}
\end{Cpp}

