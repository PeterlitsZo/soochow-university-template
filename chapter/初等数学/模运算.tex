\section{模运算} \label{sec:模运算}
这里只是用来讨论模运算的性质的。并不是用来讨论怎么进行计算。

如果想要看如何计算模下乘法,可以去看章节 \ref{sec:模下乘},如果想要看如何计算模
下的快速幂,可以去看章节 \ref{sec:快速幂}。

我们先约定 $a \equiv b \pmod n$ 和 $p \equiv q \pmod n$。

首先我们可以在模下使用任意的四则运算,除了乘法:\begin{align*}
    a + c & \equiv b + c \pmod n \\
    ac & \equiv bc \pmod n \\
    a^c & \equiv b^c \pmod n \\
    a + p & \equiv b + q \pmod n \\
    ap & \equiv bq \pmod n
\end{align*}

但是需要注意的是,如果是 $a^p \mod n$ 这种形式,下列等式可能并不满足:\begin{align*}
    a^p & \equiv b^q \pmod n \\
    c^p & \equiv c^q \pmod n
\end{align*}
这种情况我们需要参考欧拉定理或者扩展欧拉定理(见章节 \ref{sec:欧拉定理})来进行
降幂。
