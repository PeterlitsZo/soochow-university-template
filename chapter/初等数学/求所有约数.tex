\section{求所有约数}
约数个数的上限是 $2 \sqrt N$。

\subsection{试除法}
和 \ref{sec:判断素数} 检测是否是素数类似,我们可以求取所有的约数。返回一个将亡值
,所以我们可以使用一个右值引用啦。我们可以在 $O(\sqrt n)$ 的复杂度来求出所有的约
数。
\begin{Cpp}
vector<int> get_divs(int x){
    vector<int> res;
    for(int i=1;i*i<=n;i++) if(x%i==0){
        res.push_back(i);
        if(i!=x/i)res.push_back(x/i);
    }
    sort(res.begin(),res.end());
    return res;
}
\end{Cpp}


\subsection{类筛法}
我们可以使用类似筛法的方法来求出所有的约数。如下:
\begin{Cpp}
inline void get_factor(){
    for(int i = 1; i <= N; i ++) {
        for(int k = 1; k <= N; k += 1) {
            factor[i].push_back(i);
            // factor[i][++factor[i][0]] = i;
        }
    }
}
\end{Cpp}


\subsection{基于线性素数筛的快速求法}
基于章节 \ref{sec:欧拉筛} 中的欧拉筛先筛出所有素数,再求。

复杂度就是约数的个数。在 $10^8$ 的范围内不会超过 $10^3$。
\begin{Cpp}
// ...
int SMFA[P_RAG],PS[P_RAG];
void PS_init();

int divs[7000],divs_len;
// 依赖于欧拉筛产生的最大质因子。
void fast_divs(int x){
  divs_len=1;
  divs[0]=1;
  while(x>1){
    int fa=SMFA[x],cnt=0;
    while(x%fa==0){
      x/=fa,cnt++;
    }
    int bf_len=divs_len;
    for(int c=1,cc=fa;c<=cnt;c++,cc*=fa){
      for(int i=0;i<bf_len;i++){
        divs[divs_len++]=cc*divs[i];
      }
    }
  }
}
\end{Cpp}
