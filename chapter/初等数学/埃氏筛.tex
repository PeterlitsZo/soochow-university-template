\section{埃氏筛}
\label{sec:埃氏筛}



欧拉筛见章节 \ref{sec:欧拉筛}。与欧拉筛不同的是,埃氏筛的时间复杂度略大一些。埃
氏筛的复杂度为 $\FnO(n \lg \lg n)$(根据质数分布定理和调和级数可以搞出来)。

代码如下:
\begin{Cpp}
{
  // constexpr int LEN=/* 略大于 n */;
  // static bitset<LEN> notP;
  // int PS[LEN],n=/* 筛的范围 */;

  for(int i=2;i<=n;i++){
    if(!notP[i]){
      PS[++PS[0]]=i;
      for(int j=2*i;j<=n;j+=i){
        notP[j]=1;
      }
    }
  }
}
\end{Cpp}



\subsection{区间筛}
在区间比较窄的情况下,我们可以用区间筛来把区间内所有的质数筛出来。首先区间的左边
是 $L$,其次它的宽度是 $W$。

然后我们观察,那么里面所有的合数,肯定都存在一个因子(设为 $f$)是满足表达式 $f
\le \sqrt{L + W}$ 的。那么我们只需要遍历 $[2, \sqrt{L + W}]$,并筛掉区间内的所有
不对的数即可。
\begin{Cpp}
{
  // constexpr int LEN=/* 略大于 W 和 sqrt(L+W+5) */;
  // static bitset<LEN> notP,notPOff;
  // static int P[LEN];
  // int L=/* 筛的起始点 */, W=/* 筛的范围大小 */;
  // // 如果 W=R-L+1; 其中 R 是筛的终点(含),
  // // 那么对应的,for(i=0;i<W;i++) notPOff[i];

  // 初始化:
  // notP.reset(),notPOff.reset(),P[0]=0;


  for(ll i=2;i*i<=L+W+5;i++){
    if(!notP[i]){
      P[++P[0]]=i;
      for(ll j=i*2;j*j<=L+W+5;j+=i){
        notP[j]=1;
      }
      for(ll j=max((L+i-1)/i,2ll)*i;j<=L+W+1;j+=i){
        notPOff[j-L]=1;
      }
    }
  }

  // notPOff[i]==false 表示 L+i 是素数。
}
\end{Cpp}

