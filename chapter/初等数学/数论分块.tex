\section{数论分块}
\label{sec:数论分块}



数论分块主要是用来处理形如
$$
    \sum^{n}_{i=1} \lfloor{n \over i}\rfloor
$$
的式子。比如说如果 $n = 12$ 的话,那么它推导的数列就为:$\{12, 6, 4, 3, 2, 2, 1,
1, 1, 1, 1, 1\}$,我们可以看到在 $\sqrt n$ 之后很明显的分块现象 --- 在接下来的
$n - \sqrt n$ 个数中一共就只有 $\sqrt n$ 种数字。很明显,如果我们知道了每一个数
字对应的出现个数,那么我们就能够得到它的和了。

值的注意的是,数论分块要解决的问题和类欧几里德(见章节 \ref{sec:类欧几里德})类似
。

我们可以说,如果起始位置是 $l$,那么它的结束位置的下标为 $\left\lfloor\frac{n}{
\left\lfloor{n \over l}\right\rfloor}\right\rfloor$。比如对于 $n = 12$ 而言,数
字 $2$ 是从第 $5$ 个开始一直延续到第 $6$ 个。

我们不妨令这个数为 $x$,那么它满足:$xl \le n < (x + 1)\cdot l$,所以 $\lfloor{n
\over l}\rfloor$ 就表示了这个块对应的数字,而它的结束位置 $r$,首先是整数,其次
满足:$rx \le n \land (r+1)\cdot x > n$,那么自然有:$r = \lfloor{n \over
x}\rfloor = \left\lfloor\frac{n}{\left\lfloor{n \over
l}\right\rfloor}\right\rfloor$。

样例代码如下:
\begin{Cpp}
{
  ll result=0;
  for(ll i=1,nxt;i<=N;i=nxt){
    nxt=N/(N/i)+1;
    result+=(nxt-i)*(N/i);
  }
}
\end{Cpp}

值的一提的是,如果数论分块包含了多个式子相乘,比如说 $\sum^{n}_{i=1} \lfloor{n
\over i}\rfloor\lfloor{m \over i}\rfloor$ 等之类的,我们可以跳转到 $min(f(n, l),
f(m, l))$ 即可。
