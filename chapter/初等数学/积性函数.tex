
\section[积性函数]{积性函数\footnote{参考了很多 bilibili 上的 BV1H4411D7qo 视频
和 OI-wiki。}}\label{sec:积性函数}
积性函数分为普通的积性函数和完全积性函数。

积性函数的定义如下:对于函数 $f$ 而言,满足下式的即为积性函数:
$$
\begin{cases}
    f(1)  = 1,                                 \\
    f(pq) = f(p) \cdot f(q), p, q \text{互质}. \\
\end{cases}
$$

而完全积性函数而言,即使 $p$ 和 $q$ 不互质,式子 $f(pq) = f(p) \cdot f(q)$ 也是
恒成立的。

举例来说,欧拉函数 $\varphi(n)$,莫比乌斯函数 $\mu(n)$,$k$ 一定时的最大公约数
$\gcd(n, k)$,这些都是积性函数\footnote{目前好像还没有证明欧拉函数。看了一些证明
,都是循环论证,气死爷了。}。

举例来说,如果 $k=12$,那么 $\gcd(60, 12) = \gcd(12, 12) \cdot \gcd(5, 12) = 12$
。如果用算术基本定理而言的话,应该是很容易证明的。

而对于完全积性函数,我们同样可以举例:恒等函数 $1(n) = 1$,元函数 $\epsilon(n) =
[n = 1]$,和单位函数 $\mathrm{Id}(n)$。


\subsection{积性函数的性质}
如果一个 $f(x)$ 和 $g(x)$ 均为积性函数,那么我们知道以下的函数 $h(x)$ 也均为积性函数:
\begin{align*}
    h(x) = {} & f(x^p) \\
    h(x) = {} & f^p(x) \\
    h(x) = {} & f(x)g(x) \\
    h(x) = {} & \sum_{d \mid x} f(d)g({x \over d}) = (f * g)(x)
\end{align*}

前面三个有脑子就知道,关键是最后一个的推导,其中我们令 $a$ 和 $b$ 互质,然后满足
式子 $ab = x$:
\begin{align*}
    h(a)h(b) = {} & (f*g)(a) \cdot (f*g)(b) \\
             = {} & \sum_{d_1 \mid a} f(d_1)g({a \over d_1}) \cdot \sum_{d_2
                    \mid b} f(d_2)g({b \over d_2}) \\
             = {} & \sum_{d_1 \mid a} \sum_{d_2 \mid b} f(d_1)g({a \over d_1})
                    \cdot f(d_2)g({b \over d_2})\\
             = {} & \sum_{\mathclap{d_1 \mid a, d_2 \mid b}} f(d_1 d_2) g({ab \over d_1
                    d_2}) \\
             = {} & \sum_{d \mid x} f(d) g({ab \over d})
\end{align*}



