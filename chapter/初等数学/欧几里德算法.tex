\section{欧几里德算法 \& GCD \& LCM}
\label{sec:欧几里德算法}

在一般的环境中,我们可以使用 \cmd{\_\_gcd(x, g)} 直接求解,其代码约为:
\begin{Cpp}
int gcd(int a,int b){return a?gcd(b%a,a):b;}
\end{Cpp}

欧几里德同时也求出了 $ax + by$ 这个线性组合的最小解。为了得到 $x$ 和 $y$ 的值,
也可以使用更高级的扩展欧几里德(见章节 \ref{sec:扩展欧几里德})。

而对于最大公约数,它有几个性质:\[
    ab = \gcd(a, b) \cdot \lcm(a, b)
\] \[
    abc = {\lcm(a, b, c) \gcd(a, b) \gcd(a, c) \gcd(b, c) \over \gcd(a, b, c)}
\]

同时还有一些有意思的性质:
\begin{itemize}
    \item 如果 $a$ 是奇数且 $b$ 是偶数,那么 $a \mid b \Rightarrow a \mid {b
        \over 2}$。
\end{itemize}



\subsection{扩展欧几里德}
\label{sec:扩展欧几里德}\label{subsec:扩展欧几里德}

简单的欧几里德求最大公约数可以参考 \ref{sec:欧几里德算法} 章中的相关代码。扩展欧
几里德不仅可以得到最大公约数,并且可以得到斐蜀定理(见章节 \ref{sec:裴蜀定理})
中提到的 $x$ 和$y$ 的值。其代码如下:
\begin{Cpp}
int exgcd(int a,int b,int& x,int& y){
  if(!a) return x=0,y=1,b;
  int res=exgcd(b%a,a,y,x);
  x-=(b/a)*y;
  return res;
}
\end{Cpp}



\subsubsection{解同余方程}

也就是说,我们可以利用扩展欧几里德来求出:\[
    ax + by = (a,  b)
\]
但是很明显解出来的 $x$ 和 $y$ 并不是唯一解。那么所有解形成的解空间是什么呢?可以
证明出来,$(X, Y) = (x + t {b \over (a, b)}, y - t {a \over (a, b)})$,证明见下。

这很容易证明。对于 $ax + by = (a, b)$ 和 $aX + bY = (a, b)$,那么有:\[
    \begin{aligned}
    & a(X - x) + b(Y - y) = 0 \\
    \Longrightarrow & {a \over (a, b)} (X - x) = - {b \over (a, b)} (Y - y)
    \end{aligned}
\]

而 $\left({a \over (a, b)}, {b \over (a, b)}\right) = 1$,所以自然 ${b \over (a,
b)} \mid (X - x)$,故 $X = x + {b \over (a, b)} t$。

那么自然 $X$ 的最小正整数解为 $x \bmod {b \over (a, b)}$。

当然如果我们求解的是:\[
    ax + by = d
\]这个在 $(a, b) \mid d$ 的时候比较好办,如果不满足的话,自然也没有解了。



\subsubsection{求解逆元}

我们也可以使用扩展欧几里德来求解逆元:如果模为 $b$,而我们要求 $a$ 在模 $b$ 意义
下的逆元的话,那么我们可以使用 \cmd{exgcd(a,b,x,y)},如果返回的结果是 $1$ 的话,
那么说明它们满足:\[
    ax + by = 1
\]

变换到模意义下有:\[
    ax \equiv 1 \pmod b
\]

那么 $x$ 即为 $a$ 在模 $b$ 意义下的逆元。如果 $a$ 和 $b$ 互质的话,我们当然可以
用欧拉定理(见章节 \ref{sec:欧拉定理})来求解逆元,不过这是吃力不讨好的事情。如
果 $b$ 本身是一个素数的话,那么我们可以直接使用费马小定理(见章节 \ref{sec:费马%
小定理})。



