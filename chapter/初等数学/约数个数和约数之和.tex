
\section{约数个数和约数之和}
根据算术基本定理,一个数 $x$ 可以表示为:$x = p_1^{c_1} \times p_2^{c_2} \times
\ldots p_k^{c_k}$。

那么约数个数为:$(c_1 + 1) \times (c_2 + 1) \times \ldots \times (c_k + 1)$;同
理,约数之和为 $(p_1^0 + \ldots + p_1^{c_1}) \times \ldots \times (p_k^0 +
\ldots + p_k^{c_k})$。其中约数个数的上限是 $2\sqrt N$。实际中 $10^8$ 以内约数个
数最多的数的约数个数也小于 $10^3$。

我们在 \ref{sec:分解质因数} 章,分解质因数中类似的方法,我们可以在 $O(\sqrt n)$
的时间复杂度内得到算术基本定理对应的良好的解。自然我们可以在 $O(\sqrt n)$ 的复杂
度内得到对应的解。


