\section{斐波那契数}
\label{sec:斐波那契}

组合数学的斐波那契运用见章节 \ref{sec:组合数学斐波那契}。

斐波那契数有几个性质:
\begin{enumerate}
    \item 存在一个斐波那契数是另外两个斐波那契数的乘积。
\end{enumerate}

\subsection{通项公式}
斐波那契数满足:
\[
    \begin{bmatrix}
        F_{n+2} \\ F_{n+1}
    \end{bmatrix} =
    \begin{bmatrix}
        1 & 1 \\
        1 & 0 \\
    \end{bmatrix}
    \begin{bmatrix}
        F_{n+1} \\ F_n
    \end{bmatrix}
\]

而我们有:
\[
    \begin{bmatrix}
        F_2 \\ F_1
    \end{bmatrix} =
    \begin{bmatrix}
        1 \\ 1
    \end{bmatrix}
\]

那么有:
\[
    \begin{bmatrix}
        F_n \\ F_{n-1}
    \end{bmatrix} = 
    \begin{bmatrix}
        1 & 1 \\
        1 & 0 \\
    \end{bmatrix} ^{n-2}
    \begin{bmatrix}
        1 \\ 1
    \end{bmatrix}
\]

这里矩阵的幂,我们可以参考 \ref{sec:矩阵的幂} 章节中提到的方法来求解矩阵的幂。不
过我们也可以使用矩阵快速幂来做它。

所以我们有:\[
    F_n = {\sqrt 5 \over 5} \left[ \left({1 + \sqrt 5 \over 2}\right)^n +
    \left({1 - \sqrt 5 \over 2}\right)^n \right]
\]
