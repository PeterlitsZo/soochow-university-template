\section[循环节]{循环节\protect\footnote{这里大部分参考了
\url{https://blog.csdn.net/bogedaye/article/details/109658038}。李如一的视频,非
常不错,由浅入深,十分好用。}}
\label{sec:循环节}

我们只研究两种循环序列。一种是 $A = \{a \times i \bmod n\}$,一种是 $A = \{a^i
\bmod n\}$。

它们肯定会从某一个节点开始进入循环。判断序列开头是否也在循环中也很重要。

根据纯循环和混循环的定义(见章节 \ref{sec:纯循环和混循环}),我们可以知道 $\{a
\times i \bmod n\}$ 模型的一定是纯循环,即给定第 $i$ 个元素 $A_i$ 的前驱为
$A_{i-1} = (A_i - a) \bmod n$。同理$\{a^i \bmod p\}$ 一定是纯循环(因为在模 $p$
下 $a$ 一定有逆元)。

但是 $\{a^i \bmod n\}$ 而言,$(a, n)$ 可能不等于 $1$,这也说明了它可能没有对应的
逆元。而找不到逆元,说明存在一个元素可能有两个前驱。比如说 $4^i \bmod 4$ 得到的
序列就是$\{0, 1, 1, 1, \ldots\}$。



\subsection{$a \times i \bmod n$}
对于一个序列 $A = \{a \times i \bmod n\}$,它的循环节长度是 $n \over (a, n)$。

那么我们可以知道,如果 $a \times x \equiv b \pmod n$,那么对于 $x \in [0, n)$ 这
个范围内,如果 $(a, n) \mid b$,那么有解。如果 $b \nmid (a, n)$,那么它没有解。

这样子,$A$ 全部都是 $(a, n)$ 的倍数。而 $[0, n)$ 中正好有 $n \over (a, n)$ 个其
倍数。因为它是纯循环,所以它的循环节长度就是 $n \over (a, n)$。



\subsection{$a^i \bmod n$}
\subsubsection{$n$ 是素数的情况}
我们先研究弱化情况,即:$A = \{a^i \bmod p\}$。这会让我们想起费马小定理(见章节
\ref{sec:费马小定理}),即 $a^{p-1} \equiv 1 \pmod p$。这说明循环节最多是 $p - 1$。
因为它也是纯循环,所以循环节是 $p - 1$ 的一个约数。

我们可以这么来进行记忆:对于 $a \neq 0$,我们有 $a^i \not\equiv 0 \pmod p$(因为
$a$ 再怎么乘,也不可能是 $p$ 的倍数),那么它最多只会轮询 $[1, p-1]$ 共 $p - 1$
个数。这个记忆方法没什么道理,但是告诉了我们循环节是 $p - 1$。


\subsubsection{$(a, n) = 1$ 的情况}
扩展的普通情况,即:$A = \{a^i \bmod n\}$。我们首先注意到有 $A_i = A_{i-1}
\times a \bmod n$,那么如果 $A$ 中连续出现了两个相同的数,那么它一定会循环。

如果 $(a, n) = 1$,那么我们有 $(a^i \bmod n, n) = 1$。那么 $A$ 中最多也只有
$\varphi(n)$ 个不同的数。

根据这个我们可以推导出欧拉定理(见章节 \ref{sec:欧拉定理}),即 $a^{\varphi(n)}
\equiv 1 \pmod n, (a, n) = 1$。


\subsubsection{普遍情况}
现在抛掉所有的附加条件。那么按理 $A$ 它可能对应着一个混循环,也可能对应一个纯循
环。但是我们在一开始已经提出了一个反例,那么它不可能是一个纯循环,这说明了它一定
是一个混循环。

先别慌,我们可以假设它有条件 $(A_i, n) = (A_{i+1}, n) = d$,那么对于 $a_i = {A_i
\over d}$ 而言,在模 $n \over d$ 的情况下,$a_i$ 就能退化到之前 $(a, n) = 1$ 的
情况,这个时候它的循环节长度就是$\varphi({n \over d})$。

好吧,上面的普遍情况一点也不普遍。我们需要一个 $(A_i, n)$ 稳定的条件才能推出这个
。那要是不成立呢?我们用反证法尝试证明一下。

首先我们有发现 $(A_{i-1}, n) \mid (A_i, n)$。

而它一定是一个混循环,也就是说一定在某一个时候发生了 $(A_{i-1}, n) = (A_i, n)$,
而 $(A_{i-1}, n) = (A_i, n)$ 能推出来 $(A_i, n) = (A_{i+1}, n)$。

如图 \ref{fig:a^i mod n 的混循环} 所示,混循环的角角,其长度最长为 $\lg n$,而混
循环的循环节长度,一定是 $\varphi(n \over d)$,也同时是 $\varphi(n)$ 的一个约数。

\begin{figure}
\centering
\begin{tikzpicture}
    \foreach \angle / \label in {210/c, 180/d, 150/e, 120/f, 90/g, 60/h, 30/i, 0/j,
    330/k, 300/l, 270/m} {
        \node[TNode] (\label) at (\angle:2cm) {$\label$};
    }
    \foreach \label [count=\count] in {b, a} {
        \node[TNode] (\label) at (210:2cm) [xshift=0.57 * \count cm, yshift=-1 * \count cm] {$\label$};
    }
    \foreach\nxt[count=\nxti, remember=\nxt as \bef (initially c)] in {d, e, f, g, h, i, j, k, l, m, c} {
        \draw[->] (\bef) to[bend left=15] (\nxt);
    }
    \foreach \bef / \nxt in {a/b, b/c} {
        \draw[->] (\bef) to (\nxt);
    }
\end{tikzpicture}
\caption{$a^i \bmod n$ 的混循环}
\label{fig:a^i mod n 的混循环}
\end{figure}

根据这个,我们就可以推导出扩展欧拉定理(见章节 \ref{subsec:扩展欧拉定理}),那么
有\[
    a^i \equiv \begin{cases}
        a^i,                & i \le \varphi(n) + T, \\
        a^{i - \varphi(n)}, & i > \varphi(n) + T.
    \end{cases}
\]


\subsection{求解模方程}
\subsubsection{方程 $a \times x \equiv b \pmod n$}
这个还比较简单,我们先看看它有没有解。如果有解,那么一定有 $(a, n) \mid b$,反之
如果 $(a, n) \nmid b$,那么它一定没有解。

这样的话,那么我们可以利用扩展欧几里德(见章节 \ref{subsec:扩展欧几里德})解出来
。

\subsubsection{方程 $a^x \equiv b \pmod n$}
朴素思想是利用扩展欧拉定理(见章节 \ref{subsec:扩展欧拉定理})来降幂,这样子复杂
度就是 $\FnO(\varphi(n))$。

但是我们还有更加优秀的解法!那就是 BSGS 算法(见章节 \ref{sec:BSGS})!哇哈哈。

但是上面的适用于 $(a, n) = 1$ 的情况,如果不互素的话,那么它就会很难。
