\section{欧拉函数}
\label{sec:求欧拉函数} \label{sec:欧拉函数}



欧拉函数的定义是:小于等于 $n$ 的所有与 $n$ 互质的数的个数。

这也说明了:
\begin{itemize}
    \item $\varphi(p) = p-1$。
    \item $\varphi(p^i) = p^{i-1}(p - 1)$。
    \item 因为 $\varphi$ 是一个积性函数,所以 $\varphi(pq) = \varphi(p)
        \varphi(q)$(其实这里可能有循环论证的嫌疑,实际上这里可以利用章节
        \ref{sec:中国剩余定理} 的中国剩余定理来证明)。
\end{itemize}

众所周知,欧拉函数为 $\varphi(x) = x \prod^{n}_{i=1}(1-\frac{1}{p_i})$。同时我们可
以知道 $\phi(1) = 1$,而对于素数 $p$ 而言有 $\phi(p) = p - 1$,故其代码为:
\begin{Cpp}
int phi(int x){
    int res=x;
    for(int i=2;i*i<=x;i++)if(x%i==0){
        res=res/i*(i-1);
        while(x%i==0)x/=i;
    }if(x>1)res=res/x*(x-1);
    return res;
}
\end{Cpp}

可以看出其代码和 \ref{sec:分解质因数} 章分解质因数类似,其复杂度故也为 $O(\sqrt
n)$。

这里我们可以根据欧拉函数来解决欧拉定理(见章节 \ref{sec:欧拉定理})。



\subsection{线性筛求欧拉函数}
\label{subse:线性筛球欧拉函数} \label{sec:线性筛求欧拉函数}

我们可以用类似 \ref{sec:欧拉筛} 的算法来求出所有数的欧拉函数值。这是因为欧拉函数
有着一个性质:当 $p$ 与 $x$ 互质时,有满足式子:$\phi(p x) = \phi(p) \times
\phi(x)$,具有这种性质的函数被叫做积性函数。而不互质的时候呢,恰好欧拉函数有对应
的性质:$\phi(p x) = p \times \phi(x)$

代码如下:
\begin{Cpp}
const int RAG=/* PS 这个素数表和欧拉函数表的最大范围 */;
int PS[RAG/* 因为素数分布定理,P_RAG 等于 10^P 的时候,右移 log_2 P 位即可*/];
int elr[RAG]
bitset<RAG> notP;

void init() {
    notP[0]=notP[1]=true;
    elr[0]=-1,elr[1]=1;
    for(int i=2;i<P_RAG;i++){
        if(!notP[i])
            PS[++PS[0]]=i,elr[i]=i-1;
        for(int pi=1;PS[pi]*i<P_RAG;pi++){
            notP[PS[pi]*i]=true;
            if(i%PS[pi]==0){
                elr[PS[pi]*i]=PS[pi]*elr[i];
                break;
            }else
                elr[PS[pi]*i]=elr[PS[pi]]*elr[i];
        }
    }
}
\end{Cpp}

其中我们认为,如果 $\phi(p \cdot x)$ 中的 $x \bmod p$,这说明 $p \cdot x$ 有的所
有素因数 $x$ 都有,既有:$\phi(p \cdot x) = p \times x \times \prod^{s}_{i=1}
\frac{p_i - 1}{p_i} = p \times \phi(x)$。反之,我们有:$\phi(p \cdot x) = (p -
1) \times \phi(x)$。

即,我们把它分为两种方案来求值:1. 互素,我们采用积性函数的性质。2. 非互素,我们
采用他本身的性质。


