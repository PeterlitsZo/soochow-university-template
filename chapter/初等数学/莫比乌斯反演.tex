\section{莫比乌斯反演}
\label{sec:莫比乌斯反演}



需要前置知识积性函数(见章节 \ref{sec:积性函数})和狄利克雷卷积(见章节
\ref{sec:狄利克雷卷积})。

首先我们需要证明莫比乌斯函数 $\mu(n)$ 满足式子:\[
    \sum_{d \mid n} \mu(d) = [n = 1]
\]

我们知道莫比乌斯函数为:\[
    \mu(n) = \begin{cases}
        1,      & n = 1, \\
        0,      & n \text{ 含有平方因子}, \\
        (-1)^k, & k \text{ 为 } n \text{ 的本质不同的质因子个数}.
    \end{cases}
\]

假设我们已经证明完了,那么根据狄利克雷卷积,我们可以把它转换为:\[
    \mu * 1 = \epsilon
\]

而对应的,公式:\[
    F(n) = \sum_{d \mid n} f(d)
\]

能转换为:\begin{align*}
    f * 1        = {} & F \\
    f * 1 * \mu  = {} & F * \mu \\
    f * \epsilon = {} & F * \mu \\
    f            = {} & F * \mu
\end{align*}

综上,有:\[
    f(n) = \sum_{d \mid n} \mu(d) F({n \over d})
\]

也就是说,在函数集合和狄利克雷卷积形成的代数结构是一个群:它满足封闭性、结合律、
单位元(元函数 $\epsilon$),和逆元(恒等函数 $1$ 的逆元就是莫比乌斯函数 $\mu$)
。

举例来说,我们令 $f(p)$ 用来表示在 $1 \sim N$ 的范围内,$\gcd(x, y) = p$ 的对数
,那么 $F(p) = (1 * f)(p)$ 则可以用来表示 $p \mid \gcd(x, y)$ 的个数。很明显第二
个更容易解得。那么乘上逆元,我们有:\begin{align*}
    f(p) = {} & (\mu * F)(p) \\
         = {} & \sum_{p \mid d} \mu({d \over p}) F(d) \\
         = {} & \sum_{d = 1}^{p} \mu(d){N \over dp}\cdot{N \over dp}
\end{align*}



\subsection{$\varphi * \mathbf{1} = \mathrm{id}$}
$\varphi * \mathbf{1} = \mathrm{id}$ 作为莫比乌斯反演中一个重要的结论,我们根据
函数 $\varphi$ 是一个积性函数,那么我们只需要证明出 $(\varphi * \mathbf{1})(n' =
p^c) = n'$ 即可:\begin{align*}
    (\varphi * \mathbf{1})(n') = {} & \sum_{d \mid n'} \varphi(d) \\
                               = {} & \sum_{i = 0}^{c} \varphi(p^i) \\
                               = {} & 1 + \sum_{i = 1}^{c} p^{i - 1} \cdot (p - 1) \\
                               = {} & p^c = n'
\end{align*}

如果两边都卷上一个 $\mu$,那么我们就有了 $\varphi = \mathrm{id} * \mu$。



\subsection{案例一:$\gcd$ 的和}
\paragraph{题目} 求 \[
    \sum_{i=1}^{n} \sum_{j=i+1}^{n} \gcd(i, j)
\]

\paragraph{推导} 首先我们令 $d = \gcd(i, j)$,枚举之:\begin{align*}
     & \sum_{d=1}^{n} d \sum_{i=1}^{n} \sum_{j=i+1}^{n} [\gcd(i, j) = d]\\
    =& \sum_{d=1}^{n} d \sum_{i=1}^{\lfloor{n \over d}\rfloor}
    \sum_{j=i+1}^{\lfloor{n \over d}\rfloor} [\gcd(i, j) = 1] \\
    =& \sum_{d=1}^{n} d \sum_{i=1}^{\lfloor{n \over d}\rfloor}
    \sum_{j=i+1}^{\lfloor{n \over d}\rfloor} \sum_{e \mid \gcd(i, j)} \mu(e)
\end{align*}

现在我们又开始枚举 $e$ 了啦!我们知道 $e$ 同时是 $i$ 和 $j$ 的公约数。也就是说,
在给定的范围内,有多少个 $\langle i, j \rangle$ 同时是都是 $e$ 的倍数。

我们知道,$1$ 到 $\left\lfloor{n \over d}\right\rfloor$ 范围内,有
$\left\lfloor{n \over de}\right\rfloor$ 个数是 $e$ 的倍数,从中随便徐选两个,于
是有:\begin{align*}
    & \sum_{d=1}^{n} d \sum_{e=1}^{n} \mu(e) { \left\lfloor{n \over
        de}\right\rfloor \left(\left\lfloor{n \over de}\right\rfloor - 1\right)
        \over 2}
\end{align*}

我们敏锐地注意到,如果 $D = de$,那么 $D > n$ 时就没有意义。我们根据这个来减少枚
举范围,有:\begin{align*}
    & \sum_{D=1}^{n} \sum_{d \mid D} d \mu({D \over d}) {\LRfloor{n \over
        D}\LRbra{\LRfloor{n \over D} - 1} \over 2} \\
    = & \sum_{D=1}^{n} {\LRfloor{n \over D} \LRbra{\LRfloor{n \over D} - 1} \over
        2} \varphi(D) 
\end{align*}

\paragraph{扩展} 求 \[
    \sum_{i=1}^n \sum_{j=1}^n \gcd(i, j)
\]

解为:\[
    \sum_{D=1}^{n} \LRfloor{n \over D}^2 \varphi(D) 
\]

