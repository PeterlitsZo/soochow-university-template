\section{欧拉函数} \label{sec:求欧拉函数} \label{sec:欧拉函数}
欧拉函数的定义是:小于等于 $n$ 的所有与 $n$ 互质的数的个数。

这也说明了:
\begin{itemize}
    \item $\varphi(p) = p-1$。
    \item $\varphi(p^i) = p^{i-1}(p - 1)$。
    \item 因为 $\varphi$ 是一个积性函数,所以 $\varphi(pq) = \varphi(p)
        \varphi(q)$(其实这里可能有循环论证的嫌疑,实际上这里可以利用章节
        \ref{sec:中国剩余定理} 的中国剩余定理来证明)。
\end{itemize}

众所周知,欧拉函数为 $\varphi(x) = x \prod^{n}_{i=1}(1-\frac{1}{p_i})$。同时我们可
以知道 $\phi(1) = 1$,而对于素数 $p$ 而言有 $\phi(p) = p - 1$,故其代码为:
\begin{Cpp}
int phi(int x){
    int res=x;
    for(int i=2;i*i<=x;i++)if(x%i==0){
        res=res/i*(i-1);
        while(x%i==0)x/=i;
    }if(x>1)res=res/x*(x-1);
    return res;
}
\end{Cpp}

可以看出其代码和 \ref{sec:分解质因数} 章分解质因数类似,其复杂度故也为 $O(\sqrt
n)$。

这里我们可以根据欧拉函数来解决欧拉定理(见章节 \ref{sec:欧拉定理})。
