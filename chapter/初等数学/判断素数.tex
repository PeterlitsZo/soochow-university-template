\section{判断素数}\label{sec:判断素数}
\subsection{朴素算法}
在 $O(n)$ 的复杂度下能够判定一个数到底是素数还是合数。
\begin{Cpp}
int is_prime(int x){
    if(x<2)return 0;
    for(int i=2;i*i<=x;i++){
        if(x%i==0)return 0;
    }
    return 1;
}
\end{Cpp}

当然常数更小的是:
\begin{Cpp}
// 注意可能爆 int。
bool is_prime(int x){
    if(x==2||x==3) return true;
    if(x<=1||x%2==0||x%3==0) return false;
    for(int i=5;i*i<=x;i+=6){
        if(x%i==0||x%(i+2)==0) return false;
    }
    return true;
}
\end{Cpp}

\subsection{欧拉筛算法}
根据章节 \ref{sec:欧拉筛} 中介绍的欧拉筛,我们可以求得 $\le \sqrt n$ 的所有素数
,之后我们再根据这些素数来判断 $n$ 是不是一个素数。我们可以根据这个来压缩测试的
时间复杂度。

根据素数分布定理(见章节 \ref{sec:素数分布定理}),我们可以降低时间复杂度为原来
的 $1 \over \ln \sqrt n$。

\subsection{米勒罗宾}
米勒罗宾是一个 $\FnO(n^{0.25})$ 的优秀算法。我自己写过,但是不仅慢,而且也消失在
了历史长河中了。

从小狮子那里借的板子(快速幂见章节 \ref{sec:快速幂},因为有溢出的情况,记得换成
模下乘):
\begin{Cpp}
typedef long long ll;
//typedef __int128 lll;

inline ll fpow(ll x,ll p,ll MOD);

inline bool mr(ll x,ll b){
  ll k=x-1;
  while(k){
    ll cur=qp(b,k,x);
    if(cur!=1 && cur!=x-1)
      return false;
    if((k&1)==1 || cur==x-1)
      return true;
    k>>=1;
  }
  return true;
}

inline bool prime(ll x){
  // 468 - 562 - 482 - 559 - 81 and type - ll
  if(x==46856248255981ll||x<2)
    return false;
  if(x==2||x==3||x==7||x==61||x==24251)
    return true;
  return mr(x,2)&&mr(x,61);
}
\end{Cpp}
