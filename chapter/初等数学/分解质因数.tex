\section{分解质因数}\label{sec:分解质因数}
\subsection{朴素地分解质因数}
和 \ref{sec:判断素数} 章中很像,可以用来分解质因数。一般也会在分解中进行计算和储
数值,这里用标准输出代替。
\begin{Cpp}
{
  for(int i=2;i*i<=x;i++)if(x%i==0){
    printf("%d ",i);
    while(x%i==0)x/=i;
  }
  if(x%i==0){
    printf("%d ",i);
  }
}
\end{Cpp}



\subsection{利用素数来分解质因数}
注意,这是通过 $\FnO(\sqrt n)$ 的算法来分解得到的。如果卡常的话,我们可以使用欧
拉筛先筛出 $[1, \sqrt n]$ 内的素数,之后再用素数来在 $\FnO({\sqrt n \over \ln
n})$ 的时间复杂度内来进行分解。
\begin{Cpp}
int divs[2000],divs_len;
// 依赖于欧拉筛产生的最大质因子。
void fast_divs(int x){
  divs_len=1;
  divs[0]=1;
  while(x>1){
    int fa=SMFA[x],cnt=0;
    while(x%fa==0){
      x/=fa,cnt++;
    }
    int bf_len=divs_len;
    for(int c=1,cc=fa;c<=cnt;c++,cc*=fa){
      for(int i=0;i<bf_len;i++){
        divs[divs_len++]=cc*divs[i];
      }
    }
  }
}
\end{Cpp}



\subsection{利用魔法来分解质因数}
这也是从小狮子哪里 copy 过来的。
\begin{Cpp}
typedef long long ll;
//typedef __int128 lll;

ll quick_mul(ll x,ll y,ll mod) {
  ll ans=0;
  while(y!=0){
    if(y&1==1)ans+=x,ans%=mod;
    x=x+x,x%=mod;
    y>>=1; 
  }
  return ans;
}

inline ll qp(ll x,ll p,ll mod){
  ll ans=1;
  while(p)
  {
    if(p&1)ans=quick_mul(ans,x,mod);
    x=quick_mul(x,x,mod);
    // x=(lll)x*x%mod;
    p>>=1;
  }
  return ans;
}

inline bool mr(ll x,ll b){
  ll k=x-1;
  while(k){
    ll cur=qp(b,k,x);
    if(cur!=1 && cur!=x-1)
      return false;
    if((k&1)==1 || cur==x-1)
      return true;
    k>>=1;
  }
  return true;
}

inline bool prime(ll x){
  if(x==46856248255981ll || x<2)
    return false;
  if(x==2 || x==3 || x==7 || x==61 || x==24251)
    return true;
  return mr(x,2)&&mr(x,61);
}

inline ll f(ll x,ll c,ll n){
  return (quick_mul(x,x,n)+c)%n;
}

inline ll PR(ll x){
  ll s=0,t=0,c=1ll*rand()%(x-1)+1;
  int stp=0,goal=1;
  ll val=1;
  for(goal=1;;goal<<=1,s=t,val=1){
    for(stp=1;stp<=goal;++stp){
      t=f(t,c,x);
      val=quick_mul(val,abs(t-s),x);
      // val=(lll)val*abs(t-s)%x;
      if((stp%127)==0){
        ll d=__gcd(val,x);
        if(d>1)return d;
      }
    }
    ll d=__gcd(val,x);
    if(d>1)return d;
  }
}
\end{Cpp}
