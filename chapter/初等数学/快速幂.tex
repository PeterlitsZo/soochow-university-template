\section{快速幂}
\label{sec:快速幂}

矩阵快速幂见章节 \ref{sec:矩阵快速幂}。

快速幂是一个很基本的代码啦。其代码如下,如果数字太大了的话,那么需要将所有的
\cmd{int} 转换为一个 \cmd{ll},如果大得过于离谱的话,那么我们还需要使用模下的乘
法,见 \ref{sec:模下乘},来避免溢出:
\begin{Cpp}
ll fpow(ll b,ll p,/*ll MOD(注意大小写)*/){
  ll res=1;
  while(p){
    // 可能需要用到模下乘的地方
    if(p&1)res=(ll)res*b%MOD;
    p>>=1;
    // 可能需要用到模下乘的地方
    b=b*b%MOD;
  }
  return res;
}
\end{Cpp}

如果 $b$ 固定而幂 $p$ 不固定的话,我们可以先使用一个 $O(\sqrt{\max(p)})$ 的时间复杂
度来预处理,之后用 $O(1)$ 的时间复杂度来求解快速幂,即使用 BSGS(见章节
\ref{sec:BSGS}):
\begin{Cpp}
{
  int x=/* 底数的大小 */;
  const int MOD=/* 模 */,LEN=/* 保证大于 stp */;
  int B[LEN],G[LEN];
  int stp=sqrt(/* 幂的范围 */);

  B[0]=G[0]=1;
  for(int i=1;i<stp;i++){
    B[i]=1ll*B[i-1]*x%MOD;
  }
  G[1]=1ll*B[stp-1]*x%MOD;
  for(int i=2;i<=MOD/stp;i++){
    G[i]=1ll*G[i-1]*G[1]%MOD;
  }
  /* x^p == 1ll*G[p/stp]*B[p%stp]%MOD */
}
\end{Cpp}
