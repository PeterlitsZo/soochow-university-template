\section{中国剩余定理}
\label{sec:中国剩余定理}



中国剩余定理是用来求解一元线性同余方程组的算法。\textbf{要求模两两之间互素,如果
模两两之间不是互素的,请看看 exCRT(见章节 \ref{subsec:exCRT})}。

对于:\[
    \left\{\,
        \begin{aligned}
            x \equiv{}&{} a_1 \pmod {m_1}, \\
            x \equiv{}&{} a_2 \pmod {m_2}, \\
              \vdots{}&{} \\
            x \equiv{}&{} a_k \pmod {m_k}. \\
        \end{aligned}
    \right.
\]我们如果需要求出 $x$(首先需要注意的是:要不然 $x$ 有无穷多解,要不然没有解,
故中国剩余定理只给出最小解,或者说,模 $\lcm(\{m_i\})$ 下的唯一解),那么我们需
要遵守下列算法流程:
\begin{enumerate}
    \item 计算 $M = m_1 \cdot m_2 \cdot \cdots \cdot m_k$。
    \item 对于第 $i$ 个方程,计算:
        \begin{enumerate}
            \item 令 $n_i = {M \over m_i}$。
            \item 计算 $n_i$ 在 $m_i$ 下的逆元 $n_i'$,以满足 $n_i n_i' \equiv 1 \pmod
                {m_i}$。其中 $n_i$ 和 $m_i$ 必然是互质的,故必然有逆元。
            \item 计算 $c_i = n_i n_i'$。
        \end{enumerate}
    \item 解为:$x \equiv \sum_{i=1}^k a_i c_i \pmod M$。
\end{enumerate}

对于 $m_i$ 而言,有:\[
    a_i c_i = a_i n_i n_i' = a_i \cdot {m \over m_i} \cdot \left({m \over m_i}\right)'
\]在模 $m_i$ 下,$n_i'$ 是 $n_i$ 的倒数,故自然 $a_i c_i \equiv a_i \pmod {m_i}$,
而在模 $m_j, j \ne i$ 下自然没有这个道理,相反因为 $m_j \mid {m \over m_i}$,故
$a_i c_i \equiv 0 \pmod {m_j}$。得证。

参考代码如下(其中扩展欧几里德见章节 \ref{subsec:扩展欧几里德}):
\begin{Cpp}
// 其中 k 代表了 a 和 m 的长度。其中 a 是余数,而 m 是模
// 。均从 0 开始。
ll exgcd(ll a,ll b,ll &x,ll &y);

ll crt(int k,ll a[],ll m[]){
  ll M=1,ans=0;
  for(int i=0;i<k;i++) M=M*m[i];
  for(int i=0;i<k;i++) {
    ll ni=M/m[i],invni,_;
    exgcd(ni,m[i],invni,_);
    ll c=ni*invni%M;
    ans=(ans+a[i]*m[i])%M;
  }
  return (ans%M+M)%M;
}
\end{Cpp}



\subsection{CRT 的应用}
有时候我们可能需要对一个答案取模,而恰好这个模可能,不是一个质数,而是两两互不相
同的质数的乘积。那么我们可以分解分别求答案,然后用 CRT 进行合并。

当然,如果模数两两不互质的情况下,可能就 G 了。但是也并非没有办法。那么我们分情
况讨论:
\begin{itemize}
    \item 两个方程。那么我们不妨假设 $a \equiv r_1 \pmod {m_1}$ 和 $a \equiv r_2
        \pmod {m_2}$,那么我们有:\[ a = b_1 m_1 + r_1 = b_2 m_2 + r_2 \]这说明
        了 $m_1 b_1 - m_2 b_2 = r_2 - r_1$。当 $(m_1, m_2) \mid r_2 - r_1$的时候
        ,说明我们可以通过扩展欧几里德(见章节\ref{subsec:扩展欧几里德})来求出
        一组可行的 $b_1$ 和 $b_2$。

        那么综上,$a$ 满足:\[a \equiv m_1 b_1 + r_1 \pmod{\lcm(m_1, m_2)}\]
    \item 多个方程。两两地做即可。
\end{itemize}



\subsection{Garner 算法}
我们提到 CRT 的解在 $\lcm(\{n_i\})$ 下是一一对应的。这是因为 CRT 的解再分别取模
,就能得到条件,而条件只能得到唯一的解,这是一个双射关系。

所以我们可以将一个数映射到一堆数上。\[
    \left\{
        \begin{aligned}
            a \equiv{}&{} a_1 \pmod {p_1}, \\
            a \equiv{}&{} a_2 \pmod {p_2}, \\
              \vdots{}&{} \\
            a \equiv{}&{} a_k \pmod {p_k}. \\
        \end{aligned}
    \right.
\]

那么如果\[
a = x_1 + x_2 p_1 + x_3 p_1 p_2 + \ldots + x_k p_1 \ldots p_{k-1}
\]

则令 $r_{i,j}$ 为 $p_i$ 在模 $p_j$ 意义下的逆,有:\[
p_i \cdot r_{i,j} \equiv 1 \pmod{p_j}
\]

那么有:\[
    x_k=(\ldots ((a_k-x_1)r_{1,k}-x_2)r_{2,k} \ldots)r_{k-1,k} \bmod p_k
\]

代码如下:
\begin{Cpp}
for(int i=0;i<k;++i) {
  x[i]=a[i];
  for(int j=0;j<i;++j) {
    x[i]=r[j][i]*(x[i]-x[j]);
    x[i]=x[i]%p[i];
    if(x[i]<0)x[i]+=p[i];
  }
}
\end{Cpp}

虽然不太懂这玩意有啥用,但是还是抄下来了。



\subsection{exCRT}
\label{subsec:exCRT}

扩展中国剩余定理和中国剩余定理不大一样。主要是中国剩余定理,即 CRT 要求模之间两
两互素。如果不互素的情况下就难办了。

先考虑两个同余方程组:\[
    \left\{\,
        \begin{aligned}
            x \equiv{}&{} a_1 \pmod {n_1}, \\
            x \equiv{}&{} a_2 \pmod {n_2}, \\
        \end{aligned}
    \right.
\]那么有 $n_1 p + a_1 = n_2 q + a_2$,因此有 $n_1 p - n_2 q = a_2 - a_1$。我们可
以使用扩展欧几里德(见章节 \ref{subsec:exGCD})解出可行解 $p$ 和 $q$。自然 $x
\equiv n_1 p + a_1 \pmod {[n_1, n_2]}$。

再考虑多个同余方程组。上面我们把了两个同余方程变成了一个,之后我们只需要每一次消
掉一个,一直到只剩下一个就差不多得了。

代码如下:
\begin{Cpp}
ll exgcd(ll a,ll b,ll &x,ll &y);

// k 是 a 和 n 的长度。a 是余数,n 是模。保证 [n_0, n_1,
// ..., n_{k-1}] <= LONG_LONG_MAX。数组从 0 开始。
ll excrt(int k,ll a[],ll n[]){
  // 初始化模和结果。
  ll M=n[0],res=a[0];
  for(int i=1;i<k;i++){
    // assert(n[i] >= 0);
    // new_res = res + M * p = a[i] + n[i] * q
    // <=> M * p + n[i] * (-q) = a[i] - res
    ll p,q,g=exgcd(M,n[i],p,q);
    if((a[i]-res)%g!=0){
      return -1;
    }
    // 得到新的模,和新的结果。
    ll newM=M/g*n[i];
    p=(__int128)(a[i]-res)/g*p%newM;
    res=((res+(__int128)M*p)%newM+newM)%newM;
    M=newM;
  }
  return res;
}
\end{Cpp}
