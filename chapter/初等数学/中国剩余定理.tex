\section{中国剩余定理} \label{sec:中国剩余定理}

中国剩余定理是用来求解一元线性同余方程组的算法。

对于:
$$\begin{cases}
    x \equiv a_1 \pmod {n_1}, \\
    x \equiv a_2 \pmod {n_2}, \\
    \quad \vdots              \\
    x \equiv a_k \pmod {n_k}, \\
\end{cases}$$

我们如果需要求出 $x$(首先需要注意的是,要不然 $x$ 有无穷多解,要不然没有解,故
中国剩余定理只给出最小解,或者说,模 $\lcm(\{n_i\})$ 下的唯一解),那么我们需要
遵守下列算法流程:
\begin{enumerate}
    \item 计算 $n = n_1 \cdot n_2 \cdot \cdots \cdot n_k$。
    \item 对于第 $i$ 个方程,计算:
        \begin{enumerate}
            \item 令 $m_i = {n \over n_i}$。
            \item 计算 $m_i$ 在 $n_i$ 下的逆元 $m_i ^ {-1}$。
            \item 计算 $c_i = m_i m_i^{-1}$,注意不要取模。
        \end{enumerate}
    \item 解为:$x \equiv \sum_{i=1}^k a_i c_i \pmod n$。
\end{enumerate}

参考代码如下(其中扩展欧几里德见章节 \ref{subsec:扩展欧几里德}):
\begin{Cpp}
// 其中 k 代表了 a 和 r 的长度。其中 a 是余数,而
// r 是模。
ll CRT(int k,ll a[],ll r[]){
  ll n,ans=0;
  for(int i=0;i<k;i++) n=n*r[i]:
  for(int i=0;i<k;i++) {
    ll m=n/r[i],b,y;
    exgcd(m,r[i],b,y);
  }
}
\end{Cpp}

\subsection{CRT 的应用}
有时候我们可能需要对一个答案取模,而恰好这个模可能,不是一个质数,而是两两互不相
同的质数的乘积。那么我们可以分解分别求答案,然后用 CRT 进行合并。

当然,如果模数两两不互质的情况下,可能就 G 了。但是也并非没有办法。那么我们分情
况讨论:
\begin{itemize}
    \item 两个方程。那么我们不妨假设 $a \equiv r_1 \pmod {m_1}$ 和 $a \equiv r_2
        \pmod {m_2}$,那么我们有:\[ a = b_1 m_1 + r_1 = b_2 m_2 + r_2 \]这说明
        了 $m_1 b_1 - m_2 b_2 = r_2 - r_1$。当 $(m_1, m_2) \mid r_2 - r_1$的时候
        ,说明我们可以通过扩展欧几里德(见章节\ref{subsec:扩展欧几里德})来求出
        一组可行的 $b_1$ 和 $b_2$。

        那么综上,$a$ 满足:\[a \equiv m_1 b_1 + r_1 \pmod{\lcm(m_1, m_2)}\]。
    \item 多个方程。两两地做即可。
\end{itemize}

\subsection{Garner 算法}
我们提到 CRT 的解在 $\lcm(\{n_i\})$ 下是一一对应的。这是因为 CRT 的解再分别取模
,就能得到条件,而条件只能得到唯一的解,这是一个双射关系。

所以我们可以将一个数映射到一堆数上。\[
    \left\{
        \begin{aligned}
            a \equiv{}&{} a_1 \pmod {p_1} \\
            a \equiv{}&{} a_2 \pmod {p_2} \\
              \vdots{}&{} \\
            a \equiv{}&{} a_k \pmod {p_k} \\
        \end{aligned}
    \right.
\]

那么如果\[
a = x_1 + x_2 p_1 + x_3 p_1 p_2 + \ldots + x_k p_1 \ldots p_{k-1}
\]

则令 $r_{i,j}$ 为 $p_i$ 在模 $p_j$ 意义下的逆,有:\[
p_i \cdot r_{i,j} \equiv 1 \pmod{p_j}
\]

那么有:\[
    x_k=(\ldots ((a_k-x_1)r_{1,k}-x_2)r_{2,k} \ldots)r_{k-1,k} \bmod p_k
\]

代码如下:
\begin{Cpp}
for(int i=0;i<k;++i) {
  x[i]=a[i];
  for(int j=0;j<i;++j) {
    x[i]=r[j][i]*(x[i]-x[j]);
    x[i]=x[i]%p[i];
    if(x[i]<0)x[i]+=p[i];
  }
}
\end{Cpp}

虽然不太懂这玩意有啥用,但是还是抄下来了。
