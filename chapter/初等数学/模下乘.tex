
\section{模下乘}\label{sec:模下乘}
模下乘,我们可以使用 \verb|__int128| 这个黑科技:
\begin{Cpp}
ll mul(ll a,ll b,/*ll MOD*/){
    return (__int128)a*b%MOD;
}
\end{Cpp}

当然除了黑魔法,我们可以使用 OI-wiki%
\footnote{\url{https://oi-wiki.org/math/quick-pow/}} 中提到的快速乘。

我们知道 \verb|unsigned long long| 在一般机子上的大小为 8 个字节,为 64 位,故有
:$$\begin{aligned}
    a \times b \bmod m ={}& a \times b - \lfloor{ab \over m}\rfloor \times m \\
        ={}& \left(a \times b - \lfloor{ab \over m}\rfloor \times m\right) \bmod 2^{64} \\
\end{aligned}$$

那么目前我们只需要算出 $\lfloor{ab \over m}\rfloor$ 即可。

我们知道 \verb|long double| 的长度比 \verb|long long| 还多一倍的长度(其中
\verb|long double| 有 16 个字节,而 \verb|long long| 有 8 个字节)。在一系列我觉
得有漏洞的逻辑证明之后,模下乘法变成了:
\begin{Cpp}
ll mul(ll a,ll b,/*ll MOD*/){
    unsigned ll c=(unsigned ll)a*b
                 -(unsigned)((long double)a/m*b+0.5L)*MOD;
    if(c<m)return c;
    return c+m;
}
\end{Cpp}

综上,还是 \verb|__int128| 靠谱一点点。


