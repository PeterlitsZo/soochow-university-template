\section{矩阵快速幂}
\label{sec:矩阵快速幂}

快速幂见章节 \ref{sec:快速幂}。矩阵快速幂和快速幂有着相同的思想。不过什么时候会
用到,那就需要经验了。

斐波那契可以用矩阵快速幂整出来。

恬不知耻地搞了别人的代码:
\begin{Cpp}
struct Matrix{
  ll m[15][15];
  
  Matrix(){
    memset(m,0,sizeof(m));
    for(int i=0;i<=n+1;i++){
      m[i][n+1]=1;
      if(i==n+1) continue;
      m[i][0]=10;
      for(int j=1;j<=i;j++)
        m[i][j]=1;
    }
  }
  
  void clear(){
    memset(m,0,sizeof(m));
    for(int i=0;i<=n+1;i++)
      m[i][i]=1;
  }
  
  void display(){
    cout<<"Matrix's begin:"<<endl;
    for(int i=0;i<=n+1;i++)
      for(int j=0;j<=n+1;j++)
        if(j<n+1) cout<<m[i][j]<<" ";
        else cout<<m[i][j]<<endl;
    cout<<"Matrix's end:"<<endl;
  }
  
  friend Matrix operator*(Matrix a,Matrix b){
    Matrix ans;
    for(int i=0;i<=n+1;i++)
      for(int j=0;j<=n+1;j++){
        ans.m[i][j]=0;
        for(int k=0;k<=n+1;k++)
          ans.m[i][j]+=a.m[i][k]*b.m[k][j]%mod;
      }
    return ans;
  }
  
  friend Matrix operator^(Matrix base,ll k){
    Matrix ans;
    ans.clear();
    while(k){
      if(k&1) ans=ans*base;
      base=base*base;
      k>>=1;
    }
    return ans;
  }
};
\end{Cpp}
