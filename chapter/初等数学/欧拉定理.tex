\section{欧拉定理} \label{sec:欧拉定理}
首先我们需要知道,欧拉定理一般用在给模下幂降次用的。普通的欧拉定理需要 $(a, n) =
1$,而扩展欧拉,虽然没有普通的欧拉那么漂亮,但是却能涉及到 $(a, n) \ne 1$ 的情况
。

它,和费马小定理(见章节 \ref{sec:费马小定理})类似,但是使用面更广,如果
$\gcd(a, m) = 1$,那么有$a^{\varphi(m)} \equiv 1 \pmod m$。

也就是说,如果 $(a, m) = 1$,那么 $a^{\varphi(m) - 1}$ 就是 $a$ 在模 $m$ 下的逆
元(突然感觉,章节 \ref{sec:扩展欧几里德} 中的扩展欧几里德好像!没错!扩展欧几里
德也可以求这种情况下的逆元!不过很明显,exgcd 的复杂度更低)。

我们可以参考章节 \ref{sec:求欧拉函数} 中的方法来求解欧拉函数。

欧拉定理可以利用循环节来进行证明(见章节 \ref{sec:循环节})。

\subsection{扩展欧拉定理} \label{subsec:扩展欧拉定理}
在欧拉定理中,我们需要一个 $(a, m) = 1$ 的条件。但是!扩展欧拉定理就不相同了!我
们同样可以用循环节(见章节 \ref{sec:循环节})的知识来进行证明~

结论如下:\[
    a^b \equiv \begin{cases}
        a^b,                                     & b < \varphi(m), \\
        a^{(b \bmod \varphi(m)) + \varphi(m)},   & b \ge \varphi(m).
    \end{cases}
    \pmod m
\]
