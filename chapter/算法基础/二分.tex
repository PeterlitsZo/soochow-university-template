\section{二分}
\label{sec:二分}

一般而言二分大部分都指的是二分答案。当然对于某一个变量 $x$ 而言,$A_x$ 随 $x$ 递
增而递增,我们需要找到第一个 $A_x$ 大于等于给定值的代码如下:
\begin{Cpp}[texcl]
ll l=/* begin */,r=/* end */,h=/* value */;

while(l<r){
  ll mid=(l+r)/2;
  ll res=/* value for mid */;

  if(res<h){ // \dingcircle{1}
    l=mid+1;
  }else{
    r=mid;
  }
}
\end{Cpp}

总结如下:
\begin{itemize}
    \item 在递增序列中,我们需要找第一个大于等于的,那么就保持 \dingcircle{1} 代
        码不变。
    \item 在递增序列中,我们需要找第一个大于的,那么就令 \dingcircle{1} 代码为
        \cmd{res<=h}。
    \item 在递减序列中,我们需要找第一个小于等于的,那么就令 \dingcircle{1} 代码
        为 \cmd{res>h}。
    \item 在递减序列中,我们需要找第一个小于的,那么就令 \dingcircle{1} 代码
        为 \cmd{res>=h}。
\end{itemize}

当然也有一些二分指的是在有序数组中二分搜索。那个可以使用 \cmd{lower\_bound} 或者
\cmd{upper\_bound}。这些搜索函数见章节 \ref{subsec:查找}。
