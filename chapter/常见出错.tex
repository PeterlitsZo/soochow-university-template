\chapter{答题技巧及常见出错}





\section{答题技巧}
\begin{itemize}
    \item 如果正着不好思考的话,那么就倒着想。
    \item 随机算法和二分答案也可以是你的朋友。
    \item 朴素的思想,也可以转化为计算贡献的思路。
    \item 倍增很有意思,有时候可以想想。
\end{itemize}





\section{一般出错}



\subsection{WA - 答案错误}
\begin{itemize}
    \item 是否超出范围了?将 \verb|int| 改成 \verb|ll| 或者 \verb|__int128|,将
        \verb|double| 改成 \verb|ld| 试试?甚至可以分开记录分子分母?还要记得检
        查检查函数的传参的类型哦。
    \item 数组的大小是否不太够啊?数组有没有正确地初始化?
    \item 是否忘记初始化了?对于会重复调用的部分,那个 \verb|static| 不可声明即
        初始化哦。
    \item 是不是右移的时候,字面量的类型范围不够大啊?比如说 \cmd{1ll<<k} 而不
        是 \cmd{1<<k}。
    \item 是否使用了已经 \verb|erase| 的迭代器?
    \item 输入输出是不是对的?尤其是多测试的样例。
\end{itemize}



\subsection{TLE - 运行超时}
\begin{itemize}
    \item 使用 \verb|bitset| 或者 \verb|bool[]| 来代替 \verb|int[]|。一般而言
        \verb|bitset| 的常数时间复杂度约为 \verb|bool[]| 的一半。
    \item \verb|map| 的键是 \verb|tuple|、\verb|pair| 之类的话,可以用类字符串哈
        希卡常过去。
    \item 做模拟题的是否可能会卡住,是否可以不用 loop 来模拟,而是在一开始计算出
        所有值,之后再来判断?
    \item 是否没有删掉 \verb|fflush(stdout)| 等开销大的,又没啥用的函数?
    \item 如果数据结构的开销太大了,必须搞一个数组卡过去。
    \item 如果搜索的时候卡住了,检查一下有没有带记忆化?或者尝试自底向上。
    \item \verb|for| 循环的时候,如果本意是递减循环的话,看看是不是实际实现时却
        写成了递增?
    \item 超出范围,有时候也会运行超时。
    \item 是不是数组开太大了?加上 \verb|static| 或者把它放到外面去。
    \item 是不是数组开小了?正如上所示,超出范围会运行超时,但是有时候是本身数组
        就开小了。
    \item \cmd{memset} 的开销太大了,减少 \cmd{memset} 的范围。
\end{itemize}



\subsection{RE - 运行时错误}
\begin{itemize}
    \item 是否使用了已经 erase 的迭代器?
    \item 是否 \cmd{--} 了 \cmd{begin} 迭代器或者 \cmd{++} 了 \cmd{end} 迭代器?
    \item 是否本来 \cmd{for} 应该递减的,结果却用了 \cmd{i++}。
\end{itemize}





\section{计算几何}
\begin{itemize}
    \item 对于斜率来说,$a \over b$ 和 $-a \over -b$ 都是一个东西。如果使用
        \verb|pair| 来保存的话,记得令某一个一定为非负数。
    \item 使用 \cmd{acos} 和 \cmd{asin} 的时候,记得限定范围。而我们还可以使用
        \cmd{atan2(\nam{x}, \nam{y})} 来代替 \cmd{atan(\nam{x/y})}。

        对于 \cmd{acos} 和 \cmd{asin} 而言,我们可以使用 \cmd{min} 和 \cmd{max}
        函数来将它们限定在 $[-1, 1]$ 的范围内,以避免得到 \cmd{NaN}:
        \cmd{min(max(\nam{x},-1.),1.)}。
\end{itemize}
