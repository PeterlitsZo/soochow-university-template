\section{Dijkstra} \label{sec:dijkstra}
Dijkstra 一般用来求解最短路问题。其中边不能为负边的情况下,我们能在 $\FnO(n
\log m)$ 的时间复杂度内进行求解。

另一个求最短路的算法,见 \ref{sec:Bellman-Ford算法} 中的 Bellman-Ford 算法。

\begin{Cpp}
struct edge{int v,w;};
struct node{
  int dis,v;
  bool operator>(const node& other)const
    {return dis>other.dis;}
};

template<typename T>
using pri_queue_big=priority_queue<
  T,vector<T>,greater<T>>;
pri_queue_big<node> Q;

dis[s]=0;
Q.push({0,s});
while(!Q.empty()){
  int v=Q.top().v; Q.pop();
  if(vis[v])continue;
  vis[v]=1;
  for(auto e:MP[v]){
    if(dis[v]+e.w < dis[e.v]) {
      dis[e.v]=dis[v]+e.w;
      Q.push({dis[e.v], e.v});
    }
  }
}
\end{Cpp}


