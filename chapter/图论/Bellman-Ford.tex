\section{Bellman-Ford 算法} \label{sec:Bellman-Ford算法}
使用动态规划来求取最短路。不同的是,它不要求边权非负。

另一个求最短路的算法,见 \ref{sec:dijkstra} 中的 Dijkstra 算法。

谈到边权可以为负,那么就不得不谈到负权环:它是一个权的和为负数的一个环,这也说明
了,如果在这个环中循环的话,那么距离就会一直向负无穷发散,那么就不存在一个有效的
最短路。

它的算法,和 \ref{sec:dijkstra} 中的 Dijkstra 算法略有不同,即 Dijkstra 算法会在
未确认的 $V$ 集合中,基于非负边的决策环境中贪心的选个距离最小的来加入 $U$ 集合中
,而 Bellman-Ford 算法不同,它不进行贪心,就纯暴力松弛,这使得它可以处理负边和发
现负权环。


