\section{Tarjan 算法}\label{sec:Tarjan}
\subsection{用于求强连通分量的 Tarjan 算法}
关于连通性相关的知识,见 \ref{sec:连通性相关} 章节 --- 连通性相关。

在了解 Tarjan 之前,我们需要了解 DFS 生成树,见 \ref{subsec:DFS生成树}。

如果要求强连通分量的话,我们可以在一个图上跑一跑 DFS 生成树。我们可以知道,如果
我们以 $u$ 作为根来跑在一个强连通分量上,该分量的其他点都在搜索树中 $u$ 的子树上
。

Tarjan 为每一个节点都定义了两个变量:\verb|dfn| 和 \verb|low|,分别表示在 DFS 过
程中搜索的次序,和能够最早回溯的{\bfseries 已经}在栈中的节点。

在 DFS 中,如果当前在节点 $u$,下一个节点是 $v$,那么有三种情况:
\begin{enumerate}
    \item 如果 $v$ 未能访问,那么用 $v$ 的 \verb|low| 值来更新 $u$ 的 \verb|low|
        值。

    \item 如果 $v$ 已经访问过了而且在栈中,那么我们考虑用 $v$ 的 \verb|dfn| 值来
        更新 $u$的 \verb|low| 值。

    \item 如果 $v$ 已经访问过了,不过没有在栈中,那么我们不做处理。
\end{enumerate}

板子如下:
\begin{Cpp}
// C++ Version
int dfn[N], low[N], dfncnt, s[N], in_stack[N], tp;
int scc[N], sc;  // 结点 i 所在 SCC 的编号
int sz[N];       // 强连通 i 的大小
void tarjan(int u) {
  low[u] = dfn[u] = ++dfncnt, s[++tp] = u, in_stack[u] = 1;
  for (int i = h[u]; i; i = e[i].nex) {
    const int &v = e[i].t;
    if (!dfn[v]) {
      tarjan(v);
      low[u] = min(low[u], low[v]);
    } else if (in_stack[v]) {
      low[u] = min(low[u], dfn[v]);
    }
  }
  if (dfn[u] == low[u]) {
    ++sc;
    while (s[tp] != u) {
      scc[s[tp]] = sc;
      sz[sc]++;
      in_stack[s[tp]] = 0;
      --tp;
    }
    scc[s[tp]] = sc;
    sz[sc]++;
    in_stack[s[tp]] = 0;
    --tp;
  }
}
\end{Cpp}
