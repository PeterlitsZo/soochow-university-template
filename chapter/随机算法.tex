\chapter{随机算法}

\section{基础}
\subsection{随机数算法}
一般而言随机数算法都是使用的 \cmd{rand()},它位于 \cmd{cstdlib} 的头文件中。在
Linux 中它的范围是 $[0, 2^{31} - 1]$。它比较慢。它需要种子,如:
\cmd{srand(time(0))}。

我们或许需要更好的随机数算法了:位于 \cmd{random} 的 \cmd{mt19937} 和
\cmd{mt19937\_64}。我们可以这么用它:
\begin{Cpp}
mt19937 R(time(0));
printf("%ud",R());
\end{Cpp}
