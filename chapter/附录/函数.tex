\section{函数}



\subsection{buildin 系列}
\cmd{\_\_builtin\_popcount} 可以用来计算一个数在二进制情况下有多少个 $1$。



\subsection{\protect\cmd{memset}} \label{sec:memset的使用} \label{subsec:memset}
我们对数组 \cmd{A} 使用:
\begin{Cpp}
memset(A, 0, sizeof A);
\end{Cpp}

来使它全部初始化为 $0$。我们也可以使用下列代码来令所有的位都为 $1$,不过值的注意
的是,在带符号数中,它是 -1 而并非最大正整数。
\begin{Cpp}
memset(A, 255, sizeof A);
\end{Cpp}

我们可以使用下列代码来取得尽可能大的正整数,但是值的注意的是,它并不是最大的整数
,并且相加的时候可能会有溢出。
\begin{Cpp}
memset(A, 127, sizeof A);
\end{Cpp}

为了避免溢出,我们可以再次右移一位,使用 \cmd{63}:
\begin{Cpp}
memset(A, 63, sizeof A);
\end{Cpp}

著名的 \cmd{0x3f3f3f3f} 的内存结构是:
\begin{Cpp}
| 00111111 | 00111111 | 00111111 | 00111111 |
\end{Cpp}

这和使用 \cmd{63} 来 \cmd{memset} 的结果相同。

而 \cmd{-63} 的内存结构是:
\begin{Cpp}
| 11000000 | 11000000 | 11000000 | 11000001 |
\end{Cpp}

所以我们可以使用下列 \cmd{memset} 来得到足够小的负数(不过明明 \cmd{-64} 按理来
说也可以,应该是为了记忆方便吧):
\begin{Cpp}
memset(A, -63, sizeof A);
\end{Cpp}



\subsection{随机}
\subsubsection{随机数}
一般而言随机数算法都是使用的 \cmd{rand()},它位于 \cmd{cstdlib} 的头文件中。在
Linux 中它的范围是 $[0, 2^{31} - 1]$。它比较慢。它需要种子,如:
\cmd{srand(time(0))}。

我们或许需要更好的随机数算法了:位于 \cmd{random} 的 \cmd{mt19937} 和
\cmd{mt19937\_64}。我们可以这么用它:
\begin{Cpp}
mt19937 R(time(0));
printf("%ud",R());
\end{Cpp}

\subsubsection{随机排列}
使用 \cmd{shuffle(\nam{begin}, \nam{end}, \nam{randomer})} 可以用来随机排列,不过
这个函数需要我们传入一个随机器。为了避免传入带来的心智负担,我们可以使用带有随机
器的随机排列函数:\cmd{random\_shuffle}。它不需要第三个参数。

不过,\cmd{\nam{randomer}} 也可以选择 \cmd{mt19937(time(NULL))}。



\subsection{查找}
\label{subsec:查找}

\subsubsection{二分查找}
在已经排序中的数组中,可以在 $\FnO(n\lg n)$ 的时间复杂度内使用二分查找查找到制定
值。一般而言,我们有两个函数用于二分查找:第一个是能返回第一个大于等于查找值的值
的对应迭代器,即 \cmd{lower\_bound};第二个能返回第一个大于查找值的值的对应迭代
器,即\cmd{upper\_bound}。它们两个迭代器的差,即为查找值的个数。



\subsection{下一个排列组合}
我们可以使用 \cmd{next\_permutation} 来得到下一个排列组合数。

在使用之前需要先排个序。



\subsection{输入输出}
在 C 中我们使用 \cmd{scanf} 和 \cmd{printf} 来进行输入输出。

\subsubsection{函数 \protect\cmd{scanf}}
它接受的格式为 \cmd{\%[*][width][length]type}。

其中 \cmd{*} 符号可以用来读入然后再跳过赋值。\cmd{width} 本身是一个数字,表示从
输入中最多拿几个用来解析。\cmd{length} 可以是 \cmd{l} 或者 \cmd{L},用来扩展
\cmd{type} 的。

如果包含空格,那么 \cmd{scanf} 会读入尽可能多的空格然后抛弃掉。

