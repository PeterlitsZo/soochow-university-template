\section{字符串处理}



\subsection{输入一行}
在清除掉之前所有的回车之后,我们可以使用 \verb|getline| 来得到一行。它有两种表示
方法,分别是 C++ 式的:
\begin{Cpp}
string S;
getline(cin,S);
\end{Cpp}

和 C 式的:
\begin{Cpp}
char *S; size_t n;
geline(&S,&len,stdin);
\end{Cpp}

其中 \cmd{n} 不是字符串的长度,而是它分配的字节数。并且会包含行末回车。



\subsection{字符串比较}
在 C 语言中,我们可以使用 \cmd{strcmp} 来进行字符串的比较。如果返回值为 \cmd{0}
,那么说明这两个字符串相等。

在 C++ 中则比较简单:使用重载过的的 \cmd{==} 操作符即可。但是不能用来比较两个
\cmd{char *} 变量就是了。



\subsection{拆分字符串}
在 C 语言中,我们可以使用 \cmd{sscanf} 来从一个字符串中进行格式化输入。
\begin{Cpp}
sscanf(const char *str,const char *format,...);
\end{Cpp}

如果拆不动,会啥都不干,所以记得在拆之前让第一个字符为 \cmd{0}。



\subsection{解析}
使用 \cmd{strtol} 来进行从字符串到数字的转换:
\begin{Cpp}
strtol(const char *str,char **end,int base);
\end{Cpp}


