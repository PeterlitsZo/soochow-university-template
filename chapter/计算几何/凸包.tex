\section{凸包}
凸包的定义是:所有内角均在 $[0, \pi]$ 的范围内的多边形为凸多边形,而对于给定点集
,能包含所有点的最小凸多边形为这个点集的凸包。

凸包在周长上一定是最优的。

\subsection{Andrew 求凸包} \label{subsec:Andrew求凸包}
首先先排序,那么我们可以知道第一个和最后一个点均在凸包上。我们逆时针走,应该均是
向左转。如果向右转,那么很明显说明当前点不在凸包上。

我们可以维护上下凸壳。

代码如下:
\begin{Cpp}
vector<vector<int>>
outerTrees(vector<vector<int>>& trees) {
  vector<vec> ps;
  for(auto &tree:trees) {
    ps.push_back({tree[0],tree[1]});
  }
  sort(ps.begin(),ps.end());
  
  vector<int> ps_stk;
  int ps_len=ps.size();
  vector<bool> used(ps_len,false);
  // 求下凸壳,从左到右遍历,第一个元素一定是下凸
  // 壳的第一个元素。
  ps_stk.push_back(0);
  for(int i=1;i<ps_len;i++){
    // 如果不是一个凸点,那么一定是前面的错!
    while(ps_stk.size()>=2){
      vec &pre=ps[ps_stk[ps_stk.size()-1]],
        &ppre=ps[ps_stk[ps_stk.size()-2]];
      int times_res=(pre-ppre).times(ps[i]-pre);
      // 没有问题了,退出循环
      if(times_res>=0) {
        break;
      }
      // 坏了,pop_back。
      used[ps_stk.back()]=false;
      ps_stk.pop_back();
    }
    // 加入当前点
    used[i]=true;
    ps_stk.push_back(i);
  }
  // 求上凸壳,从右向左遍历,最后一个元素一定是上
  // 凸壳的第一个元素(之前已经添加)。
  for(int i=ps_len-2;i>=0;i--) {
    if(used[i]) continue;
    while(ps_stk.size()>=2) {
      vec &pre=ps[ps_stk[ps_stk.size()-1]],
        &ppre=ps[ps_stk[ps_stk.size()-2]];
      int times_res=(pre-ppre).times(ps[i]-pre);
      // 没有问题了,退出循环
      if(times_res>=0) {
        break;
      }
      // 坏了,pop_back。
      used[ps_stk.back()]=false;
      ps_stk.pop_back();
    }
    // 加入当前点
    used[i]=true;
    ps_stk.push_back(i);
  }
  vector<vector<int>> res;
  for(int i=0;i==0||i<ps_stk.size()-1;i++) {
    vec &element=ps[ps_stk[i]];
    res.push_back({element.x,element.y});
  }
  return res;
}
\end{Cpp}

\subsection{Graham 求凸包}

