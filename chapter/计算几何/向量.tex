\section{向量}
向量是同时具有大小和方向的量。

模板如下:
\begin{Cpp}
struct Vec{
  double x,y;
  Vec(double x=0.0,double y=0.0):x(x),y(y){}
  Vec operator+(Vec other){return Vec(x+other.x,y+other.y);}
  Vec operator-(Vec other){return Vec(x-other.x,y-other.y);}
  Vec operator*(double k){return Vec(x*k,y*k);}
  double len2(){return x*x+y*y;}
  double len(){return sqrt(len2());}
  double times(Vec other){return x*other.y-y*other.x;}
  Vec rotate_90(){return Vec(y,-x);}
};
\end{Cpp}

\subsection{数量积} \label{subsec:向量数量积}
数量积又称内积、点积。它的运算符号一般是一个点:`$\cdot$'。它有:
$$
    \vec a \cdot \vec b = |\vec a| |\vec b| \cos \theta
$$

因为 $\theta = 0 \Rightarrow \cos \theta = 0$,故我们常常使用数量积来断定两个直
线时候垂直。

\subsection{向量积} \label{subsec:向量向量积}
向量积又称叉积。它的运算符号一般是一个叉叉:`$\times$'。
$$
    \vec a \times \vec b = \begin{vmatrix}
        \vec i & \vec j & \vec k \\
        a_x    & a_y    & a_z    \\
        b_x    & b_y    & b_z    \\
    \end{vmatrix}
$$

一般而言我们有 $a_z = b_z = 0$,这个时候,根据 \ref{sec:行列式},我们有:
$$
    \vec a \times \vec b = (a_x b_y - a_y b_x) \vec k
$$

此外,向量积对应的代数系统有:反交换律,分配律。

向量积的模对应了平行四边形的面积大小,如果为 $0$ 的话,那么说明两个直线平行。我
们还可以根据这个正负号来在 Andrew 求凸包(见 \ref{subsec:Andrew求凸包})的时候来
判断向左转抑或是向右转。


