\section{点}
\subsection{两点和距离求第三点}
利用随机算法可以快速求出所有的点。

首先我们可以查询任意一个点到某一个未知点 $(x', y')$ 的距离,假设这个未知点在第一
象限,那么我们可以查询 $(x, 0)$ 和 $(x + 1, 0)$,为简化运算,不妨设查询到的结果
其为对应距离的平方 $a$ 和 $b$,那么自然而然 $|a - b| = 2d + 1$,其中 $d =
\min(|x' - x|, |x' - x + 1|)$,即横坐标下 $x'$ 和最近的查询点之间的距离。然后我
们可以根据 $a$ 和 $b$ 的大小关系来求解它到底是距离 $(x, 0)$ 更近一点还是 $(x +
1, 0)$。

之后我们求出横坐标,纵坐标自然也轻轻松松。
\begin{Cpp}
while(true){
    // 随机两个查询点并查询
    x=rand()%XRANGE;
    printf("%d %d\n%d %d\n",x,0,x+1,0);
    fflush(stdout);

    // 记录查询结果
    ll a,b;
    scanf("%lld%lld",&a,&b);

    // P 氏猜想:当且仅当 a == b,它们查询的是两
    // 个不同的点。
    if(a==b) continue;

    // 计算答案
    ll ax=(a<b?-1:1)*(abs(a-b)-1)/2+(a<b?x:x+1);
    ll ay=sqrt(a-1ll*(ax-x)*(ax-x));

    // P 氏猜想不对也没有关系,验证答案,不对再
    // 随机即可。
    printf("%lld %lld\n",ax,ay);
    fflush(stdout);
    scanf("%lld",&a);
    if(a==0)break;
}
\end{Cpp}

\subsection{点到直线的距离}
首先我们不妨假设在二维平面上,那么如图 \ref{fig:点到直线的距离} 所示。

\begin{figure}
    \centering
    \begin{tikzpicture}
        \filldraw (0, 0) \Tdot coordinate (P) node[above=1mm]{$P$};
        \filldraw (-2, -2) \Tdot coordinate (A) node[left=1mm]{$A$};
        \filldraw (2, -1) \Tdot coordinate (B) node[right=1mm]{$B$};
        \draw (A) -- (B);
        \draw[dashed] (A) -- (P);
        \draw pic["$\alpha$", draw, angle radius=8mm]{angle=B--A--P};
        \filldraw ($(A)!(P)!(B)$) \Tdot coordinate (P') node[below=1mm]{$P'$};
        \draw (P) -- (P');
    \end{tikzpicture}
    \caption{点到直线的距离}
    \label{fig:点到直线的距离}
\end{figure}

首先根据点积(见章节 \ref{subsec:向量数量积}),我们可以很容易得到 $\cos(
\alpha)$,然后也能得到 $\sin(\alpha)$,根据 $\sin(\alpha)$,我们能得到 $PP'$ 的
距离。

那么代码如下:
\begin{Cpp}
// dot 是点,a、b 是直线过的两点
double dot_to_line_distance(Vec dot,Vec a,Vec b){
  Vec ab=b-a,adot=dot-a;
  double cosa=adot*ab/ab.len()/adot.len();
  double sina=sqrt(1-cosa*cosa);
  return sina*adot.len();
}
\end{Cpp}
