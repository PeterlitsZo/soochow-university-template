\section{NTT - 快速数论变换}
NTT 是带模的整数多项式乘法。常数较小。

NTT 可能需要一些关于原根(见章节 \ref{sec:原根})的知识。

NTT 需要的模是满足 $p = q2^n + 1$ 的质数。一般情况下,我们会使用数字:
$$\begin{aligned}
    p = 1004535809, g = 3 \\
    p = 998244353,  g = 3 \\
\end{aligned}$$

这些质数的原根含有 $3$,这是一个比较好的性质。

根据原根的性质,那么有 $3 ^ {p - 1} \equiv 1 \pmod p$。我们可以把 FFT 中的
\verb|step| 换成 $3 ^ {p - 1 \over w}$,不妨取名为 $s$。这样子我们就可以说 $s^1$
,$s^2$ 一直到 $s^w$ 都是模 $p$ 不相同的数。

FFT 中最重要的就是快速地将系数表示转换为点值表示,那么横坐标的选择就尤为重要了,
在 FFT 中我们选择了单位复根。带来了一些可能不大必要的复杂度,这里我们就选择一些
原根的幂,就不需要涉足复数了。

对比 FFT 的分治过程。对于 $A(x)$ 有:$A(x) = A'(x^2) + xA''(x^2)$ 和 $A(-x) =
A'(x^2) - xA''(x^2)$ 对吧。而对于 NTT 而言,我们依然有这些条件。那么我们如果算出
了子结构的 $g^n$ 对应的值 $A'(g^n)$ 和 $A''(g^n)$,那么算出 $A(g^{n})$ 也不算
难吧!对于原根而言,我们有:$g^n \equiv -g^{w/2 + n} \pmod p, n \le w/2$ 这么一
个性质。因此我们也能搞出来 $A(g^{w/2+n})$。(阿西吧我也不知道我在说什么了,估计
有错误,TODO)。

代码如下:
\begin{Cpp}
int P=998244353;

int up_2p(int a){
  a--; for(int i:{1,2,4,8,16}) // 4*8 -> 2*16
    a|=a>>i;
  return a+1;
}

void ntt(int NUM[],int on){
  for(int i=1;i<LEN;i++){
    // 这 rev 玩意可以预处理。
    rev[i]=(rev[i>>1]>>1)|(i&1?(LEN>>1):0);
    if(i<rev[i])swap(NUM[i],NUM[rev[i]]);
  }
  for(int w=2;w<=LEN;w<<=1){
    // P-1 能多次被 2 整除必不可少。
    int step=qpow(3,(P-1)/w);
    for(int i=0;i<LEN;i+=w){
      // cur = 3^0
      int cur=1;
      for(int j=i;j<i+w/2;j++){
        int g=NUM[j],h=1ll*cur*NUM[j+w/2]%P;
        NUM[j]=(g+h)%P,NUM[j+w/2]=(g-h+P)%P;
        cur=1ll*cur*step%P;
      }
    }
  }
  if(on==-1)for(int i=0;i<LEN;i++)
    NUM[i]/=LEN;
}
\end{Cpp}

这个板子好像没有啥用,附一个 oi-wiki 的板子吧:
\begin{Cpp}
#include <algorithm>
#include <bitset>
#include <cmath>
#include <cstdio>
#include <cstdlib>
#include <cstring>
#include <ctime>
#include <iomanip>
#include <iostream>
#include <map>
#include <queue>
#include <set>
#include <string>
#include <vector>
using namespace std;

inline int read() {
  int x = 0, f = 1;
  char ch = getchar();
  while (ch < '0' || ch > '9') {
    if (ch == '-') f = -1;
    ch = getchar();
  }
  while (ch <= '9' && ch >= '0') {
    x = 10 * x + ch - '0';
    ch = getchar();
  }
  return x * f;
}
void print(int x) {
  if (x < 0) putchar('-'), x = -x;
  if (x >= 10) print(x / 10);
  putchar(x % 10 + '0');
}

const int N = 300100, P = 998244353;

inline int qpow(int x, int y) {
  int res(1);
  while (y) {
    if (y & 1) res = 1ll * res * x % P;
    x = 1ll * x * x % P;
    y >>= 1;
  }
  return res;
}

int r[N];

void ntt(int *x, int lim, int opt) {
  register int i, j, k, m, gn, g, tmp;
  for (i = 0; i < lim; ++i)
    if (r[i] < i) swap(x[i], x[r[i]]);
  for (m = 2; m <= lim; m <<= 1) {
    k = m >> 1;
    gn = qpow(3, (P - 1) / m);
    for (i = 0; i < lim; i += m) {
      g = 1;
      for (j = 0; j < k; ++j, g = 1ll * g * gn % P) {
        tmp = 1ll * x[i + j + k] * g % P;
        x[i + j + k] = (x[i + j] - tmp + P) % P;
        x[i + j] = (x[i + j] + tmp) % P;
      }
    }
  }
  if (opt == -1) {
    reverse(x + 1, x + lim);
    register int inv = qpow(lim, P - 2);
    for (i = 0; i < lim; ++i) x[i] = 1ll * x[i] * inv % P;
  }
}

int A[N], B[N], C[N];

char a[N], b[N];

int main() {
  register int i, lim(1), n;
  scanf("%s", &a);
  n = strlen(a);
  for (i = 0; i < n; ++i) A[i] = a[n - i - 1] - '0';
  while (lim < (n << 1)) lim <<= 1;
  scanf("%s", &b);
  n = strlen(b);
  for (i = 0; i < n; ++i) B[i] = b[n - i - 1] - '0';
  while (lim < (n << 1)) lim <<= 1;
  for (i = 0; i < lim; ++i) r[i] = (i & 1) * (lim >> 1) + (r[i >> 1] >> 1);
  ntt(A, lim, 1);
  ntt(B, lim, 1);
  for (i = 0; i < lim; ++i) C[i] = 1ll * A[i] * B[i] % P;
  ntt(C, lim, -1);
  int len(0);
  for (i = 0; i < lim; ++i) {
    if (C[i] >= 10) len = i + 1, C[i + 1] += C[i] / 10, C[i] %= 10;
    if (C[i]) len = max(len, i);
  }
  while (C[len] >= 10) C[len + 1] += C[len] / 10, C[len] %= 10, len++;
  for (i = len; ~i; --i) putchar(C[i] + '0');
  puts("");
  return 0;
}
\end{Cpp}
