\section{树状数组}
简简单单的树状数组,注意 $0$ 在树状数组中是孤儿,所以要从 $1$ 开始。

首先树状数组中下标为 $i$ 的元素,其下标 $i$ 加上 $i$ 对应的 lowbit 即是它的
parent 节点。

所以我们首先可以解出 lowbit 函数,并且顺便定义一下树状数组的数组 \verb|ta| 和数
组的长度限制 \verb|TA_LEN|:
\begin{Cpp}
constexpr int TA_LEN=/* 长度 */;
int ta[TA_LEN];

int lowbit(int n){
  return n&(-n);
}
\end{Cpp}

然后我们可以根据这个来更新(而不是 ST 表那样,连更新都更新不了,见 \ref{sec:ST表
} 章):
\begin{Cpp}
int update(int inx,int val){
  int i=inx;
  while(i<TA_LEN){
    ta[i]+=val;
    i+=lowbit(i);
  }
}
\end{Cpp}

之后我们有一个查询的函数,他用来查询 $[1, r]$ 的所有值的和:
\begin{Cpp}
int query(int r){
  int result=0;
  while(r){
    result+=ta[r];
    r-=lowbit(r);
  }
  return result;
}
\end{Cpp}
