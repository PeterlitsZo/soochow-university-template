\section{线段树} \label{sec:线段树}

曾几何时,我幻想我能逃离线段树的魔爪,现在想来,终究是错付了。正所谓十个数据结构
,九个线段树,即使只是一个小小的数学/字符串选手,也依然要面临线段树的洗礼!阿西
吧!

对于 RMQ(区间最值),我们也可以使用 ST 表(见第 \ref{sec:ST表} 章)!他们说使用
ST 表会好写很多。

\subsection{线段树的数据结构}
在竞赛中数组优先。我们使用一个 \cmd{int[MAXN]} 类型的数组 \cmd{A} 来存储叶子节点
的所有值。使用 \cmd{int[MAXN<<2]}(必须是 $4$ 倍而不是 $2$ 倍。我也不知道为什么
)的数组 \cmd{seg} 来存储所有的节点维护的数据。

\subsection{建树}
建树一般采取递归的手法,其中下份代码对应的根节点为 $0$:
\begin{Cpp}
// i: 节点对应的编号。l, r: 节点管辖的范围。
void build(int i,int l,int r){
  if(l==r){ // 叶子
    cin>>A[l];
    /* ? */ // 维护 seg 数组
    return;
  }
  // 非叶子,递归即可。
  build(i*2+1,l,l+(r-l)/2);
  build(i*2+2,l+(r-l)/2+1,r);
  calc(i,l,r); // 维护 seg 数组
}
\end{Cpp}

\subsection{维护}
对于一个节点而言,它需要维护它的值始终正确。我们可以使用函数 \cmd{calc} 来维护它
,并在建树和更新的时候调用它。

\subsection{更新}
我们可以使用 \cmd{update} 来更新它的值:
\begin{Cpp}
// x 坐标的对应值更新为了 y。
void update(int i,int l,int r,int x,int y){
  if(l==r){a[x]=y; return;}
  if(x<=l+(r-l)/2)
    update(i*2+1,l,l+(r-l)/2,x,y);
  else
    update(i*2+2,l+(r-l)/2+1,r,x,y);
  calc(i,l,r);
}
\end{Cpp}

\subsection{查询}
我们依然要传入节点对应编号、节点管辖范围。除此之外,我们还需要传入查询的范围:
\begin{Cpp}
// i:节点对应的编号。l, r:节点对应的范围。
// qL, qR:查询范围,qL >= l && qR <= r 恒成立。
int ans;
void query(int i,int l,int r,int qL,int qR){
  /* ? */ // ans 求值。
  query(i*2+1,l,l+(r-l)/2,qL,qR);
  query(i*2+2,l+(r-l)/2+1,r,qL,qR);
}
\end{Cpp}

