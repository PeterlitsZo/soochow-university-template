\section{ST 表} \label{sec:ST表}
我了解这个主要是它能解决 RMQ 问题,即区间最值问题。当然也可以尝试看看线段树(见
第 \ref{sec:线段树} 章)。不过如果不维护的话,或许 ST 表会好写很多。

ST 只能处理可重复贡献,如 $\mathrm{max}$ 或者 $\mathrm{min}$ 之类的。

所以代码如下:
\begin{Cpp}
/*
  // depth of ST: n: 1e6 -> 20+, 1e3 -> 10+
  int A[LEN],ST[LEN<<1][20],n;
*/

void get_st(){
  for(int i=0;i<n;i++){
    ST[i][0]=A[i];
  }
  for(int j=1,stp=1; j<=(int)log2(n);
      j++,stp<<=1){
    for(int i=0;i<n;i++){
      // 因为我做的题目保证正数,所以可以这么做,
      // 不然就参照附录进行初始化吧。
      // AND: 可能不是 max。
      ST[i][j]=max(ST[i][j-1],ST[i+stp][j-1]);
    }
  }
}

inline int query(int l,int r){
  int lv=log2(r-l+1);
  // AND: 可能不是 max。
  return max(ST[l][lv],ST[r-(1<<lv)+1][lv]);
}
\end{Cpp}
